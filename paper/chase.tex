\section{The Chase}

	\textbf{talk about chase as nondeterministic algorithm or deterministic implementation algorithm? or both...}

	The \emph{chase} is a function that, when given a gemoetric theory, will
	generate a set of jointly minimal models for that theory. More
	specifically, if $U$ is the set of all models obtained from an execution of
	the chase over a geometric theory $T$, for any model $\mathbb{M}$ such that
	$\mathbb{M} \models T$, there is a homomorphism from some $u \in U$ to
	$\mathbb{M}$.

	There are three types of runs of the chase:
	\begin{itemize}
	\item a non-empty result in finite time
	\item an empty result in finite time
	\item an infinite run, with possible return dependent on implementation
	\end{itemize}

	Recall that geometric formul{\ae} are of the form
	\[\forall\ (free(F_L) \cup free(F_R)) : F_L \to F_R\]
	where $free$ is the function that returns the set of all free variables for
	a given formula and all $F$ are first-order logic formul{\ae} in positive
	existential form. Also recall that a geometric formula's implication is
	implicitly universally quantified over all free variables.

	Geometric logic formul{\ae} are used by the chase because they have the
	useful property where adding any relations or domain members to a model
	that satisfies a geometric logic formula will never cause the formula to no
	longer be satisified. This is particularly helpful when trying to create a
	model that satisifies all formul{\ae} in a geometric theory.

	\subsection{Algorithm}

	\subsection{Examples}

		Define $\Sigma$ as the following geometric theory

		\begin{eqnarray}
			\top    &  \to  &  \exists\ y,z : R[y,z]                             \\
			R[x,w]  &  \to  &  (\exists\ y : Q[x,y]) \vee (\exists\ z : P[x,z])  \\
			Q[u,v]  &  \to  &  (\exists\ z : R[u,z]) \vee (\exists\ z : R[z,w])  \\
			P[u,v]  &  \to  &  \bot
		\end{eqnarray}

		The following three chase runs show the different types of results
		depending on which disjunct the algorithm attempts to satisfy when a
		disjunction is encountered

		\textbf{TODO: label these}

		\begin{tabular}{lllllll}
			$\emptyset$ & $\mapsto$ & \{ & $a,b$   & $|$ & $R[a,b]$         & \} \\
			{}          & $\mapsto$ & \{ & $a,b,c$ & $|$ & $R[a,b], Q[a,c]$ & \} \\
		\end{tabular}

		\begin{tabular}{lllllll}
			$\emptyset$ & $\mapsto$ & \{ & $a,b$   & $|$ & $R[a,b]$               & \} \\
			{}          & $\mapsto$ & \{ & $a,b,c$ & $|$ & $R[a,b], P[a,c]$       & \} \\
			{}          & $\mapsto$ & \{ & $a,b,c$ & $|$ & $R[a,b], P[a,c], \bot$ & \} \\
		\end{tabular}

		\begin{tabular}{lllllll}
			$\emptyset$ & $\mapsto$ & \{ & $a,b$                & $|$ & $R[a,b]$                                        & \} \\
			{}          & $\mapsto$ & \{ & $a,b,c$              & $|$ & $R[a,b], Q[a,c]$                                & \} \\
			{}          & $\mapsto$ & \{ & $a,b,c,d$            & $|$ & $R[a,b], Q[a,c], R[d,c]$                        & \} \\
			{}          & $\mapsto$ & \{ & $a,b,c,d,e$          & $|$ & $R[a,b], Q[a,c], R[d,c], Q[d,e]$                & \} \\
			{}          & $\mapsto$ & \{ & $a,b,c,d,e,f$        & $|$ & $R[a,b], Q[a,c], R[d,c], Q[d,e], R[f,e]$        & \} \\
			{}          & $\mapsto$ & \{ & $a,b,c,d,e,f,\ldots$ & $|$ & $R[a,b], Q[a,c], R[d,c], Q[d,e], R[f,e],\ldots$ & \} \\
			{}          & $\mapsto$ & \multicolumn{5}{l}{ \ldots } \\
		\end{tabular}
