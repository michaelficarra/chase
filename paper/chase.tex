\section{The Chase}

	The \emph{chase} is a function that, when given a gemoetric theory, will
	generate a set of jointly minimal models for that theory. More
	specifically, if $U$ is the set of all models obtained from an execution of
	the chase over a geometric theory $T$, for any model $\mathbb{M}$ such that
	$\mathbb{M} \models T$, there is a homomorphism from some $u \in U$ to
	$\mathbb{M}$.

	There are three types of runs of the chase:
	\begin{itemize}
	\item a non-empty result in finite time
	\item an empty result in finite time
	\item an infinite run, with possible return dependent on implementation
	\end{itemize}

	Recall that geometric formul{\ae} are of the form
	\[\forall\ (free(F_L) \cup free(F_R)) : F_L \to F_R\]
	where $free$ is the function that returns the set of all free variables for
	a given formula and all $F$ are first-order logic formul{\ae} in positive
	existential form.

	Geometric logic formulas are used by the chase because they have the useful
	property in that adding any relations or domain members to a model that
	satisfies a geometric logic formula will never cause the formula to no
	longer be satisified. This is particularly helpful when trying to create a
	model that satisifies all formulas in a geometric theory.

	\subsection{Algorithm}

	\subsection{Examples}
