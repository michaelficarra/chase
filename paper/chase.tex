\section{The Chase}
\label{sec:chase}

	The \emph{chase} is a function that, when given a geometric theory, will
	generate a model in the set of jointly universal models for that theory. More
	specifically, if $\mathcal{U}$ is the set of all models obtained from an
	execution of the chase over a geometric theory $\mathcal{T}$, for any model
	$\mathbb{M}$ such that $\mathbb{M} \models \mathcal{T}$, there is a homomorphism from
	some model $\mathbb{U} \in \mathcal{U}$ to $\mathbb{M}$. Note that given a
	model $\mathbb{M}$ returned by the chase there may exist a model
	$\mathbb{N}$ such that $\mathbb{N} \preceq \mathbb{M}$.

	Verifying the universality of a model or the joint universality of a set of
	models requires checking if a homomorphism exists from the universal model(s)
	to each of the infinite set of all models that satisfy the theory. It may
	not at first seem obvious that generating a universal model would be a
	computable task, but it will be shown that the chase is able to do this,
	and it will be proven that the models it returns during successful runs are
	in fact members of a set of jointly universal models.

	Geometric logic sentences are used by the chase both because they are
	natural expressions of many common applications and because they take
	advantage of the useful properties of the positive-existential sentences
	of which they are constructed. Recall that, when adding any relations or
	domain members to a model that satisfies a positive-existential sentence,
	the model will always satisfy the sentence. This is particularly helpful
	when trying to create a model that satisfies all sentences in a geometric
	theory.

	\subsection{Algorithm}
	\label{sec:chase.algorithm}

		To describe the chase algorithm succinctly, it is convenient to
		introduce the following notation. Given $\mathbb{M}$ is a model, $\mathcal{T}$ is
		a geometric theory, $\sigma$ is a sentence in $\mathcal{T}$, and $\lambda :
		free(\sigma) \to |\mathbb{M}|$ is an environment, the pair
		$(\sigma,\lambda)$ is a \emph{test}, specifically a test of
		$\mathbb{M}$ based on $\mathcal{T}$. The model $\mathbb{M}$ \emph{passes} this
		test if and only if $\mathbb{M} \models_\lambda \sigma$.

		An existentially-quantified conjunction of atomics (ECA) $E$ is defined
		inductively by
		\begin{itemize}
		\item if $E$ is an atomic, $E$ is an ECA
		\item if $E$ is $\exists\ \vec{x} : \alpha$ and $\alpha$ is an ECA, $E$ is an ECA
		\item if $E$ is $\alpha \wedge \beta$ and both $\alpha$ and $\beta$ are ECAs, $E$ is an ECA
		\end{itemize}
		For simplicity, we describe the chase assuming each sentence of the
		input theory is of the form
			\[
			\delta_0 \vee\ldots\vee \delta_n \to \gamma_0 \vee\ldots\vee \gamma_k\}
			\]
		where each $\delta_i$ and $\gamma_j$ are ECAs. The chase can take this
		form as input without loss of generality. In other words, a logical
		equivalent to any legal chase input theory can be expressed in a way
		that respects these constraints.

		Next we define a \emph{chase step}, denoted $\mathbb{M}
		\xrightarrow{(\sigma,\lambda)} \mathbb{M}'$, with $\mathbb{M'}$ being
		the result of the following algorithm applied to the model
		$\mathbb{M}$, the sentence $\sigma$, and the environment $\lambda$.

		\begin{algorithm}[H]
		\DontPrintSemicolon
		\TitleOfAlgo{chaseStep :: Model ($\mathbb{M}$) $\to$ Sentence ($\sigma$) $\to$ Environment ($\lambda$) $\to$ Model}
		let $\sigma$ be $\alpha \to \beta$ \;
		choose some disjunct $e \equiv \exists\ x_0,\ldots,x_p : c_0(\vec{a_0}) \wedge\ldots\wedge c_n(\vec{a_n})$ of $\beta$ \;
		add new elements $d_0 \ldots d_p$ to $|\mathbb{M}|$ \;
		define $\theta$ as $\lambda_{[x_0 \mapsto d_0 , \ldots , x_p \mapsto d_p]}$ \;
		add each of the facts $\{c_i(\theta\vec{a_j}) \mid 0 \le i \le n \}$ to $\mathbb{M}$ \;
		return $\mathbb{M}$ \;
		\end{algorithm}

		A chase step fails if and only if the right side of the
		implication is a contradiction.

		Now that the notion of a chase step is established, the chase can be
		defined as follows.

		\begin{algorithm}[H]
		\DontPrintSemicolon
		\TitleOfAlgo{chase :: [Sentence] ($\mathcal{T}$) $\to$ Model}
		let $\mathbb{M}$ be a model that has an empty domain and an empty set of facts \;
		\While{$\mathbb{M} \not\models \mathcal{T}$}{
			choose a test $(\sigma,\lambda)$ for which $\mathbb{M}$ fails \;
			update $\mathbb{M}$ to be the result of the chase step on $\mathbb{M}$ based on $(\sigma,\lambda)$ \;
		}
		return $\mathbb{M}$
		\end{algorithm}

		There are three types of runs of the chase:
		\begin{itemize}
		\item a set of jointly universal models is found in finite time
		\item an empty result is found in finite time
		\item an infinite run with possible return dependent on implementation
		\end{itemize}

	\subsection{Examples}

		Define $\Sigma$ as the following geometric theory.

		\begin{eqnarray}
			\label{eqn:chase1}
			\top    &  \to  &  \exists\ x,y : R(x,y)                             \\
			\label{eqn:chase2}
			R(x,y)  &  \to  &  (\exists\ z : Q(x,z)) \vee P                      \\
			\label{eqn:chase3}
			Q(x,y)  &  \to  &  (\exists\ z : R(x,z)) \vee (\exists\ z : R(z,y))  \\
			\label{eqn:chase4}
			P       &  \to  &  \bot
		\end{eqnarray}

		The following three chase runs show the different types of results
		depending on which disjunct the algorithm attempts to satisfy when a
		disjunction is encountered.

		\begin{enumerate}
		\item A non-empty result in finite time:

			\begin{tabular}{lllllll}
				$\emptyset$ & $\mapsto$ & \{ & $a,b$   & $\mid$ & $R(a,b)$         & \} \\
				{}          & $\mapsto$ & \{ & $a,b,c$ & $\mid$ & $R(a,b), Q(a,c)$ & \}
			\end{tabular}

			Since the left side of \eqref{eqn:chase1} is always satisfied, but
			its right side is not, domain members $a$ and $b$ and fact $R(a,b)$
			are added to the initially empty model to satisfy \eqref{eqn:chase1}.
			The left side of \eqref{eqn:chase2} holds, but the right side does
			not, so one of the disjuncts $\exists\ z : Q(x,z)$ or $P(x)$ is
			chosen to be satisfied. Assuming the left operand is chosen, $x$
			will already have been assigned to $a$ and a new domain member $c$
			and a new fact $Q(a,c)$ will be added to satisfy \eqref{eqn:chase2}.
			With the current model, all rules hold under any environment.
			Therefore, this model is universal.

		\item An empty result in finite time:

			\begin{tabular}{lllllll}
				$\emptyset$ & $\mapsto$ & \{ & $a,b$   & $\mid$ & $R(a,b)$               & \} \\
				{}          & $\mapsto$ & \{ & $a,b,c$ & $\mid$ & $R(a,b), P(a,c)$       & \} \\
				{}          & $\mapsto$ & \{ & $a,b,c$ & $\mid$ & $R(a,b), P(a,c), \bot$ & \} \\
				{}          & $\mapsto$ & \multicolumn{5}{l}{ $\varepsilon$ }
			\end{tabular}

			Again, domain members $a$ and $c$ and fact $R(a,b)$ are added to
			the initial model to satisfy \eqref{eqn:chase1}. This time, when
			attempting to satisfy \eqref{eqn:chase2}, the right side is chosen
			and $P$ is added to the set of facts. After adding this new fact,
			rule \eqref{eqn:chase4} no longer holds; its left side is
			satisfied, but its right side does not hold for all of the bindings
			for which it is satisfied. When we attempt to satisfy the right
			side of \eqref{eqn:chase4}, it is found to be a contradiction and
			therefore unsatisfiable. Since this model can never satisfy this
			theory, the chase fails.

		\item An infinite run:

			\begin{tabular}{lllllll}
				$\emptyset$ & $\mapsto$ & \{ & $a,b$        & $\mid$ & $R(a,b)$                                 & \} \\
				{}          & $\mapsto$ & \{ & $a,b,c$      & $\mid$ & $R(a,b), Q(a,c)$                         & \} \\
				{}          & $\mapsto$ & \{ & $a,\ldots,d$ & $\mid$ & $R(a,b), Q(a,c), R(d,c)$                 & \} \\
				{}          & $\mapsto$ & \{ & $a,\ldots,e$ & $\mid$ & $R(a,b), Q(a,c), R(d,c), Q(d,e)$         & \} \\
				{}          & $\mapsto$ & \{ & $a,\ldots,f$ & $\mid$ & $R(a,b), Q(a,c), R(d,c), Q(d,e), R(f,e)$ & \} \\
				{}          & $\mapsto$ & \multicolumn{5}{l}{ $\ldots$ }
			\end{tabular}

			Like in the example above that returned a non-empty, finite result,
			the first two steps add domain members $a$, $b$, and $c$ and facts
			$R(a,b)$ and $Q(a,c)$. The left side of the implication in
			\eqref{eqn:chase3} now holds, but the right side does not. In order
			to make the right side hold, one of the disjuncts needs to be
			satisfied. If the right disjunct is chosen, a new domain member $d$
			and a new relation $R(d,c)$ will be added. This will cause the left
			side of the implication in \eqref{eqn:chase2} to hold for $R(d,c)$,
			but the right side will not hold for the same binding. $Q(d,e)$
			will be added, and this loop will continue indefinitely unless a
			different disjunct is chosen in \eqref{eqn:chase2} or
			\eqref{eqn:chase3}.

		\end{enumerate}

	\subsection{Foundations}

		%\begin{theorem}
		%	A geometric theory $\mathcal{T}$ is satisfiable if and only if
		%	either there exists an infinite fair chase run of $\mathcal{T}$ or
		%	there exists a successful fair chase run of $\mathcal{T}$.
		%\end{theorem}

		%\begin{proof}
		%\end{proof}

		\label{fairness_definition}
		Intuitively, a deterministic realization of the chase algorithm (the
		pairing of the nondeterministic chase algorithm with an evaluation
		strategy) is \emph{fair} if the scheduler would not allow for a (rule,
		binding) pair to go unevaluated during an infinite run. The formal
		definition is below.

		\begin{definition}
		Let $\mathcal{T}$ be a geometric theory, and let
			% $\rho = <s_1, s_2, \ldots>$
			\[
			\rho \quad = \quad \mathbb{F}_0 \xrightarrow{(\sigma_0,\lambda_0)}
			\mathbb{F}_1 \xrightarrow{(\sigma_1,\lambda_1)} % \mathbb{F}_1
			\cdots \mathbb{F}_{i} \xrightarrow{(\sigma_i,\lambda_i)}
			\mathbb{F}_{i+1} \xrightarrow{(\sigma_{i+1},\lambda_{i+1})} \cdots
			\]
		be an infinite run of the chase starting with the empty model $\mathbb{F}_0$.

		Let $C$ be the set of domain elements that occur anywhere in the chase;
		that is,
			\[
			C = \bigcup \{ | \mathbb{F}_{i} | \mid 0 \leq i < \infty \}
			\]

		We say that $\rho$ is \emph{fair} if for every pair $(\sigma,\lambda)$
		such that $\sigma \in \mathcal{T}$, $\lambda$ is an environment over $C$, and
		$\mathbb{F}_i \not\models_{\lambda} \sigma$, there exists $j$ such that
		$i \leq j$ and $(\sigma,\lambda)$ is $(\sigma_j,\lambda_j)$.
		\end{definition}

		\begin{lemma}
			\label{lemma_chase_step_minimality_preservation}
			Let $\mathcal{T}$ be a geometric theory and $\mathbb{M}$ be a model of $\mathcal{T}$.
			Let $\mathbb{N}$ be a model and suppose there exists a homomorphism
			$h : \mathbb{N} \to \mathbb{M}$. If $\mathbb{N}$ fails the test
			$(\sigma,\lambda)$, there exists a chase step $\mathbb{N}
			\xrightarrow{(\sigma,\lambda)} \mathbb{N}'$ and a homomorphism $h'
			: \mathbb{N}' \to \mathbb{M}$.
		\end{lemma}

		\begin{proof}

			We give the proof in the setting without equality. See example
			\ref{simulate_equality} in section
			\ref{sec:technical_background.geometric_logic.examples} to see that
			we can treat all equalities as if they were defined as a relational
			equivalent.

			Suppose $\sigma$ is in the form
				\[
				E(\vec{x}) \to \bigvee_i F_i(\vec{x})
				\]

			Since $\mathbb{N} \not\models_\lambda \sigma$, we know $\mathbb{N}
			\models_\lambda E(\vec{x})$ while $\mathbb{N} \not\models_\lambda
			\bigvee_i F_i(\vec{x})$. Since $h$ is a homomorphism, $\mathbb{M}
			\models_{h\circ\lambda} E(\vec{x})$. Since $M
			\models_{h\circ\lambda} \sigma$, we have $\mathbb{M}
			\models_{h\circ\lambda} \bigvee_i F_i(\vec{x})$ for some disjunct
			$i$. There exists a chase step $\mathbb{N}
			\xrightarrow{(\sigma,\lambda)} \mathbb{N}'$ that will choose this
			disjunct.

			In case $F_i(\vec x)$ is of the form $\bigwedge_j R_{ij}(\vec x)$
			the chase step generates $\mathbb{N}'$ by adding the facts
			$R_{ij}(\lambda(\vec{x}))$ for each $j$. To see that $h :
			\mathbb{N} \to \mathbb{M}$ is also a homomorphism from
			$\mathbb{N}'$ to $\mathbb{M}$ it suffices to see that each of the
			new facts added to $\mathbb{N}'$ is preserved, by $h$, in
			$\mathbb{M}$. That is, we want to see that for each $j$, the fact
			$R_{ij}(h(\lambda(x_0)),\ldots,h(\lambda(x_n))$ holds in
			$\mathbb{M}$. But this follows from our earlier observation that
			$\mathbb{M} \models_{h\circ\lambda} F_i(\vec x)$.

			In case $F_i$ is of the form $\exists y_0,\ldots,y_m : \bigwedge_j
			R_{ij}(x_0,\ldots,x_n,y_0,\ldots,y_m)$, the chase step generates
			$\mathbb{N}'$ by duplicating $\mathbb{N}$ and adding new elements
			$a_0,\ldots,a_m$ to $|\mathbb{N}'|$. The chase step also adds the
			facts $R_{ij}(\lambda(x_0),\ldots,\lambda(x_n),a_0,\ldots,a_m)$ to
			$\mathbb{N}'$. There must exist some $e_0,\ldots,e_m \subseteq
			|\mathbb{M}|$ such that the fact
			$R_{ij}(h(\lambda(x_0)),\ldots,h(\lambda(x_n)),e_0,\ldots,e_m) \in
			\mathbb{M}$. Define $h'(a_g) = e_g$ for each $0 \le g \le m$ and
			$h'(k) = h(k)$ for all $k \in |\mathbb{N}|$. It follows that $h'$
			is a homomorphism $h' : \mathbb{N}' \to \mathbb{M}$ and $\mathbb{M}
			\models_{h'\circ\lambda} F_i(\vec x)$.
		\end{proof}

		The model that plays the role that $\mathbb{M}$ plays in Lemma
		\ref{lemma_chase_step_minimality_preservation} is sometimes referred to
		as the \emph{oracle}.

		\begin{theorem}
			Let $\mathcal{T}$ be a geometric theory. For any model $\mathbb{M}$ such that
			$\mathbb{M} \models \mathcal{T}$, there exists a run of the chase that
			returns a model $\mathbb{N}$ such that $\mathbb{N} \models \mathcal{T}$ and
			$\mathbb{N} \preceq \mathbb{M}$.
		\end{theorem}

		\begin{proof}
			Define $\mathbb{F}_0$ as the empty model. The empty function is a
			homomorphism $h : \mathbb{F}_0 \to \mathbb{M}$. Starting with
			$\mathbb{F}_0$, iterate Lemma
			\ref{lemma_chase_step_minimality_preservation} over the models
			generated by successive chase steps with respect to oracle
			$\mathbb{M}$ and theory $\mathcal{T}$.

			If the chase stops after $n$ chase steps, the model $\mathbb{F}_n$
			will satisfy $\mathbb{F}_n \models \mathcal{T}$ and $\mathbb{F}_n
			\preceq \mathbb{M}$.

			If the chase does not stop and the chase steps are chosen fairly,
			the generated infinite model, which we may call
			$\mathbb{F}_\infty$, will satisfy $\mathbb{F}_\infty \models
			\mathcal{T}$ and $\mathbb{F}_\infty \preceq \mathbb{M}$.

			To show that $\mathbb{F}_\infty \models \mathcal{T}$, let $\sigma$
			be a sentence of $\mathcal{T}$ of the form $\forall\ \vec{x} :
			\alpha(\vec x) \to \beta(\vec x)$. To see that $\mathbb{F}_\infty
			\models \sigma$, let $\lambda$ be a binding from $\vec x$ to
			$|\mathbb{F}_\infty|$. It suffices to show $\mathbb{F}_\infty
			\models_\lambda \alpha(\vec x) \to \beta(\vec x)$. For convenience,
			let $a_0,\ldots,a_n$ be $\lambda(x_0),\ldots,\lambda(x_n)$.

			If $\mathbb{F}_\infty \not\models_\lambda \alpha(\vec x)$, then
			$\mathbb{F}_\infty \models \sigma$ as desired. Otherwise,
			$\mathbb{F}_\infty \models_\lambda \alpha(\vec x)$. Here, it
			suffices to show that $\mathbb{F}_\infty \models_\lambda \beta(\vec
			x)$. For some fixed $i$, $a_0,\ldots,a_n \in |\mathbb{F}_i|$. By
			fairness, there exists some $j$ such that $i \le j$ and the chase
			step from $\mathbb{F}_j$ uses $\sigma$ and $\lambda$: $\mathbb{F}_j
			\xrightarrow{(\sigma,\lambda)} \mathbb{F}_{j+1}$. Because
			$\beta(\vec x)$ is in positive-existential form, $F_{j+1}
			\models_\lambda \beta(\vec x)$. Because $\mathbb{F}_{j+1}$ is a
			submodel of $\mathbb{F}_\infty$, $\mathbb{F}_{j+1} \preceq
			\mathbb{F}_\infty$. Thus, $\mathbb{F}_\infty \models_\lambda
			\beta(\vec x)$ as desired.
		\end{proof}

	\subsection{History}

		In \cite{FKMP02} \emph{Data Exchange: Semantics and Query Answering},
		Fagin et. al. first introduce a chase algorithm. The version they
		defined disallows disjunctions. Input sentences without disjunctions
		are appropriate to the database setting of \cite{FKMP02} and allow a
		completely deterministic algorithm.

		The chase was originally used to solve the problem of data exchange. As
		Fagin et. al. states, ``Data exchange is the problem of taking data
		structured under a source schema and creating an instance of a target
		schema that reflects the source data as accurately as possible". The
		solution to the stated problem was to find a universal model. In theories
		without disjunction, a single universal model exists that has a
		homomorphism to any other model that satisfies the theory. This
		absolute universal model can be calculated from a single run of their
		deterministic chase algorithm.

		The definition of the chase algorithm used by Fagin et. al. is similar
		to the one defined in section \ref{sec:chase.algorithm} in all ways
		except that it does not have to choose a disjunct.
