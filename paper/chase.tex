\section{The Chase}

	The \emph{chase} is a function that, when given a gemoetric theory, will
	generate a set of jointly minimal models for that theory. More
	specifically, if $\mathcal{U}$ is the set of all models obtained from an
	execution of the chase over a geometric theory $T$, for any model
	$\mathbb{M}$ such that $\mathbb{M} \models T$, there is a homomorphism from
	some model $\mathbb{U} \in \mathcal{U}$ to $\mathbb{M}$.

	Geometric logic formul{\ae} are used by the chase because they have the
	useful property where adding any relations or domain members to a model
	that satisfies a geometric logic formula will never cause the formula to no
	longer be satisified. This is particularly helpful when trying to create a
	model that satisifies all formul{\ae} in a geometric theory.

	There are three types of runs of the chase:
	\begin{itemize}
	\item a non-empty result in finite time
	\item an empty result in finite time
	\item an infinite run, with possible return dependent on implementation
	\end{itemize}

	\subsection{Algorithm}

	\subsection{Examples}

		Define $\Sigma$ as the following geometric theory

		\begin{eqnarray}
			\top    &  \to  &  \exists\ y,z : R[y,z]                             \\
			R[x,w]  &  \to  &  (\exists\ y : Q[x,y]) \vee (\exists\ z : P[x,z])  \\
			Q[u,v]  &  \to  &  (\exists\ z : R[u,z]) \vee (\exists\ z : R[z,w])  \\
			P[u,v]  &  \to  &  \bot
		\end{eqnarray}

		The following three chase runs show the different types of results
		depending on which disjunct the algorithm attempts to satisfy when a
		disjunction is encountered

		A non-empty result in finite time:

		\begin{tabular}{lllllll}
			$\emptyset$ & $\mapsto$ & \{ & $a,b$   & $|$ & $R[a,b]$         & \} \\
			{}          & $\mapsto$ & \{ & $a,b,c$ & $|$ & $R[a,b], Q[a,c]$ & \} \\
		\end{tabular}

		An empty result in finite time:

		\begin{tabular}{lllllll}
			$\emptyset$ & $\mapsto$ & \{ & $a,b$   & $|$ & $R[a,b]$               & \} \\
			{}          & $\mapsto$ & \{ & $a,b,c$ & $|$ & $R[a,b], P[a,c]$       & \} \\
			{}          & $\mapsto$ & \{ & $a,b,c$ & $|$ & $R[a,b], P[a,c], \bot$ & \} \\
			{}          & $\mapsto$ & \multicolumn{5}{l}{ $\varepsilon$ } \\
		\end{tabular}

		An infinite run:

		\begin{tabular}{lllllll}
			$\emptyset$ & $\mapsto$ & \{ & $a,b$         & $|$ & $R[a,b]$                                 & \} \\
			{}          & $\mapsto$ & \{ & $a,b,c$       & $|$ & $R[a,b], Q[a,c]$                         & \} \\
			{}          & $\mapsto$ & \{ & $a,b,c,d$     & $|$ & $R[a,b], Q[a,c], R[d,c]$                 & \} \\
			{}          & $\mapsto$ & \{ & $a,b,c,d,e$   & $|$ & $R[a,b], Q[a,c], R[d,c], Q[d,e]$         & \} \\
			{}          & $\mapsto$ & \{ & $a,b,c,d,e,f$ & $|$ & $R[a,b], Q[a,c], R[d,c], Q[d,e], R[f,e]$ & \} \\
			{}          & $\mapsto$ & \multicolumn{5}{l}{ $\ldots$ } \\
		\end{tabular}
