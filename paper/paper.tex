\documentclass[11pt, a4paper]{article}

\usepackage{amssymb}
\usepackage{listings}
\usepackage[nodayofweek]{datetime}
\usepackage{setspace}
\usepackage{graphicx}
\usepackage{amsmath}
\usepackage[breaklinks]{hyperref}
\usepackage[all]{hypcap}

\setlength{\textheight}{8.63in}
\setlength{\textwidth}{5.9in}
\setlength{\topmargin}{-0.2in}
\setlength{\oddsidemargin}{0.3in}
\setlength{\evensidemargin}{0.3in}
\setlength{\headsep}{0.0in}

\setlength{\parindent}{0cm}
\setlength{\parskip}{0.4cm}

\begin{document}


\begin{titlepage}
\begin{center}
	\textsc{
			{\huge Generating Minimal Models}\\[0.2cm]
			{\large for}\\[0.3cm]
			{\huge Geometric Theories}
		}\\[1cm]
	A Major Qualifying Project Report\\
	submitted to the Faculty of\\[0.7cm]
	\textsc{ \large Worcester Polytechnic Institute }\\[0.7cm]
	in partial fulfillment of\\
	the requirements for the degree of\\
	Bachelor of Science\\[1cm]
	by\\[1cm]
	~\hspace{2cm}\dotfill\hspace{2cm}~\\
	\textsc{\Large Michael Ficarra}\\[1cm]
	on\\[1cm]
	{\Large \today}\\
	\vfill
	\begin{flushright}
		\hspace{8cm}\dotfill \\
		\textsc{Daniel Dougherty}\\
		professor, project advisor\\
	\end{flushright}
\end{center}
\end{titlepage}


\pagenumbering{roman}

~\\
\vfill
\begin{abstract}
This paper describes a method, referred to as the chase, for generating jointly
minimal models for a geometric theory. A minimal model for a theory is a model
for which there exists a homomorphism to any other model that can satisfy the
theory. These models are useful in solutions to problems in many practical
applications, including but not limited to firewall configuration examination,
protocol analysis, and access control evaluation. Also described is a Haskell
implementation of the chase and its development process and design decisions.
\end{abstract}
\vfill
~\\
\newpage

\renewcommand{\contentsname}{Table of Contents}
\tableofcontents \newpage
%\listoffigures \newpage
%\listoftables \newpage

\pagenumbering{arabic}


%\onehalfspacing

% main paper sections
\section{Introduction}

	This document details work done for a Major Qualifying Project at Worcester
	Polytechnic Institute by Michael Ficarra in partial fulfillment of the
	requirements for a Bachelor of Science degree.

	\subsection{Goals}

		The two main goals of this Major Qualifying Project are:

		\begin{enumerate}
		\item first, to implement an algorithm known as ``the chase" accurately
		and with a well-defined, usable interface
		\item second, to use the chase implementation for a real-world
		application: generating models used in analysis of a specific protocol
		\end{enumerate}

		Secondary goals include implementing various optimizations and
		integrating the chase implementation into a program that can take
		advantage of the functionality it provides.

	\subsection{The Chase}

		\emph{The chase} is an algorithm used to find jointly minimal models
		for a set of geometric logic formul{\ae}. Many common real-world
		problems can be expressed as a set of geometric logic formul{\ae}.
		When these problems have an unbounded scope of possible solutions, the
		chase can be used to find the possible solutions that are interesting.
		This allows researchers to go through only models that will behave
		differently than most models, rather than testing each on separately.

		\textbf{ more introduction stuff here }

\section{Technical Background}

	\subsection{Vocabulary}

		A \emph{vocabulary} consists of a set of pairings of \emph{relation
		symbols}, with a non-negative integral arity.

		A relation symbol, often called a \emph{predicate}, can be any unique
		symbol. It is used in conjunction with an arity to refer to relations
		in models.

	\subsection{Models}

		A \emph{model} $\mathbb{M}$ for a vocabulary $\mathcal{V}$ is a construct that consists of:
		\begin{itemize}
		\item a set, denoted $|\mathbb{M}|$, called the \emph{universe} or \emph{domain} of $\mathbb{M}$
		\item for each pairing of a predicate $R$ and an arity $k$ in $\mathcal{V}$, a relation $R^\mathbb{M}_k \subseteq |\mathbb{M}|$
		\end{itemize}
		It is important to distinguish the predicate, which is just a symbol,
		from the relation that it refers to when paired with its arity. The
		relation itself is a set of tuples of members from the universe.

	\subsection{First-order Logic}

		\emph{First-order logic}, also called \emph{predicate logic}, is a
		formal logic system. A first-order logic formula is defined inductively by
		\begin{itemize}
		\item if $R$ is a relation symbol of arity $k$ and each of $x_0 \ldots x_{k-1} \in \vec{x}$ is a variable, then $R(\vec{x})$ is a formula, specifically an \emph{atomic formula}
		\item if $x$ and $y$ are variables, then $x = y$ is a formula
		\item $\top$ and $\bot$ are formul{\ae}
		\item if $\alpha$ is a formula, then $(\neg\alpha)$ is a formula
		\item if $\alpha$ and $\beta$ are formul{\ae}, then $(\alpha \wedge \beta)$ is a formula
		\item if $\alpha$ and $\beta$ are formul{\ae}, then $(\alpha \vee \beta)$ is a formula
		\item if $\alpha$ and $\beta$ are formul{\ae}, then $(\alpha \to \beta)$ is a formula
		\item if $\alpha$ is a formula and $x$ is a variable, then $(\forall x : \alpha)$ is a formula
		\item if $\alpha$ is a formula and $x$ is a variable, then $(\exists x : \alpha)$ is a formula
		\end{itemize}

		For our purposes, this logic system will not contain any constant
		symbols or function symbols, which are commonly included in first-order
		logic. A function $f(x,y)$ can be encoded as the geometric logic
		formula $F(x,y,z1) \wedge F(x,y,z2) -> z1 = z2$.  Constant symbols can
		be encoded as relations with an arity of zero, for example $C()$.

		A shorthand notation may sometimes be used which omits either the left
		or right side of an implication and denotes ($\top \to \sigma$) and
		($\sigma \to \bot$) respectively.  If $\alpha$ is a formula and
		$\vec{x}$ is a set of variables of size $k$, then $(\forall \vec{x} :
		\alpha)$ is $(\forall x_0 \ldots \forall x_{k-1} : \alpha)$.  If
		$\alpha$ is a formula and $\vec{x}$ is a set of variables of size $k$,
		then $(\exists \vec{x} : \alpha)$ is $(\exists x_0 \ldots \exists
		x_{k-1} : \alpha)$.

	\subsection{Variable Binding}

		The set of free variables in a formula is defined inductively as follows
		\begin{itemize}
		\item any variable occurring in an atomic formula is a free variable
		\item the set of free variables in $\top$ and $\bot$ is $\emptyset$
		\item the set of free variables in $x = y$ is $\{x,y\}$
		\item the set of free variables in $\neg\alpha$ is $free(\alpha)$
		\item the set of free variables in $\alpha \wedge \beta$ is $free(\alpha) \cup free(\beta)$
		\item the set of free variables in $\alpha \vee   \beta$ is $free(\alpha) \cup free(\beta)$
		\item the set of free variables in $\alpha \to    \beta$ is $free(\alpha) \cup free(\beta)$
		\item the set of free variables in $\forall x : \alpha$ is $free(\alpha) - \{x\}$
		\item the set of free variables in $\exists x : \alpha$ is $free(\alpha) - \{x\}$
		\end{itemize}
		A formula $\alpha$ is a \emph{sentence} if $free(\alpha) = \emptyset$.

	\subsection{Environment}

		An \emph{environment} $\lambda$ for a model $\mathbb{M}$ is a function
		from a variable $v$ to a domain member $e$ where $e \in |\mathbb{M}|$.
		The syntax $\lambda_{[v \mapsto a]}$ denotes the environment
		$\lambda'(x)$ that returns $a$ when $x=v$ and returns $\lambda(x)$
		otherwise.

	\subsection{Satisfiability}

		A model $\mathbb{M}$ is said to satisfy a formula $\sigma$ in an
		environment $\lambda$, denoted $\mathbb{M} \models_\lambda \sigma$ and
		read ``under $\lambda$, $\sigma$ is true in $\mathbb{M}$", when
		\begin{itemize}
		\item $\sigma$ is a relation symbol $R$ and $R(\lambda(a_0) , \ldots , \lambda(a_n)) \in \mathbb{M}$ where $a$ is a set of variables
		\item $\sigma$ is of the form $\neg\alpha$ and $\mathbb{M} \not\models_\lambda \alpha$
		\item $\sigma$ is of the form $\alpha\wedge\beta$ and both $\mathbb{M} \models_\lambda \alpha$ and $\mathbb{M} \models_\lambda \beta$
		\item $\sigma$ is of the form $\alpha\vee\beta$ and either $\mathbb{M} \models_\lambda \alpha$ or $\mathbb{M} \models_\lambda \beta$
		\item $\sigma$ is of the form $\alpha\to\beta$ and either $\mathbb{M} \not\models_\lambda \alpha$ or $\mathbb{M} \models_\lambda \beta$
		\item $\sigma$ is of the form $\forall x : \alpha$  and for every $x' \in |\mathbb{M}|$, $\mathbb{M} \models_{\lambda[x \mapsto x']} \alpha$
		\item $\sigma$ is of the form $\exists x : \alpha$  and for at least one $x' \in |\mathbb{M}|$, $\mathbb{M} \models_{\lambda[x \mapsto x']} \alpha$
		\end{itemize}
		The notation $\mathbb{M} \models \sigma$ (no environment specification)
		means that, under the empty environment $l$, $\mathbb{M} \models_l \sigma$.

		A model $\mathbb{M}$ satisfies a set of formul{\ae} $\Sigma$ under an
		environment $\lambda$ if for every $\sigma$ such that $\sigma \in
		\Sigma$, $\mathbb{M} \models_\lambda \sigma$. This is denoted as
		$\mathbb{M} \models_\lambda \Sigma$ and read ``$\mathbb{M}$ is a model
		of $\Sigma$".

	\subsection{Entailment}

		Given an environment $\lambda$, a set of formul{\ae} $\Sigma$ is said
		to \emph{entail} a formula $\sigma$ ($\Sigma \models_\lambda \sigma$)
		if the set of all models satisfied by $\Sigma$ under $\lambda$ is a
		subset of the set of all models satisfying $\sigma$ under $\lambda$.

		The notation used for satisfiability and entailment is very similar, in
		that the operator used ($\models$) is the same, but they can be
		distinguished by the type of left operand.

	\subsection{Homomorphisms}

		A \emph{homomorphism} from $\mathbb{A}$ to $\mathbb{B}$ is a function
		$h: |\mathbb{A}|\to|\mathbb{B}|$ such that, for each relation symbol
		$R$ and tuple $\langle a_0 , \ldots , a_n \rangle$ where $a \subseteq
		|\mathbb{A}|$, $\langle a_0 , \ldots , a_n  \rangle \in R^\mathbb{A}$
		implies $\langle h(a_0) , \ldots , h(a_n) \rangle \in R^\mathbb{B}$.

		A homomorphism $h$ is also a \emph{strong homomorphism} if, for each
		relation symbol $R$ and tuple $\langle a_0 , \ldots , a_n \rangle$
		where $a \subseteq |\mathbb{A}|$, $\langle a_0 , \ldots , a_n  \rangle
		\in R^\mathbb{A}$ if and only if $\langle h(a_0) , \ldots , h(a_n)
		\rangle \in R^\mathbb{B}$.

		The notation $\mathbb{M} \preceq \mathbb{N}$ means that there exists a
		homomorphism $h : \mathbb{M} \to \mathbb{N}$. The identity function is
		a homomorphism from any model $\mathbb{M}$ to itself.  Homomorphisms
		have the property that $\mathbb{A} \preceq \mathbb{B} \wedge \mathbb{B}
		\preceq \mathbb{C}$ implies $\mathbb{A} \preceq \mathbb{C}$.

		However, $\mathbb{M} \preceq \mathbb{N} \wedge \mathbb{N} \preceq
		\mathbb{M}$ does not imply that $\mathbb{M} = \mathbb{N}$, but instead
		that $\mathbb{M}$ and $\mathbb{N}$ are \emph{homomorphically
		equivalent}.  For example, fix two models $\mathbb{M}$ and $\mathbb{N}$
		that are equivalent except that $\mathbb{N}$ has one more domain
		member than $\mathbb{M}$. Both $\mathbb{M} \preceq \mathbb{N}$ and
		$\mathbb{N} \preceq \mathbb{M}$ are true, yet $\mathbb{M} \neq
		\mathbb{N}$. Homomorphic Equivalence between a model $\mathbb{M}$ and a
		model $\mathbb{N}$ is denoted $\mathbb{M} \simeq \mathbb{N}$.

		Given models $\mathbb{M}$ and $\mathbb{N}$ where $\mathbb{M} \preceq
		\mathbb{N}$ and a formula in positive-existential form $\sigma$, if
		$\mathbb{M} \models \sigma$ then $\mathbb{N} \models \sigma$.

		A homomorphism $h : \mathbb{A} \to \mathbb{B}$ is also an
		\emph{isomorphism} when $h$ is 1:1 and onto and the inverse function
		$h^{-1} : \mathbb{B} \to \mathbb{A}$ is a homomorphism.

	\subsection{Minimal Models}

		Minimal models, also called \emph{universal} models, are models for a
		theory $T$ with the special property that there exists a homomorphism from
		the minimal model to any other model that satisfies $T$.  Intuitively,
		minimal models have no unnecessary entities or relations and thus
		display the least amount of constraint necessary to satisfy the theory
		for which they are minimal.

		A set of models $\mathcal{M}$ is said to be \emph{jointly minimal} for
		a set of formul{\ae} $\Sigma$ when every model $\mathbb{N}$ such that
		$\mathbb{N} \models \Sigma$ has a homomorphism from a model $\mathbb{M}
		\in \mathcal{M}$ to $\mathbb{N}$.

		More than one minimal model may exist for a given theory. Given a model
		$\mathbb{M}$ that is minimal for a theory $T$, any model $\mathbb{N}$
		such that $\mathbb{N} \simeq \mathbb{M}$ is also minimal for $T$.

		Not every theory must have a minimal model. A simple example of this is
		the theory containing a single formula $\sigma$ where $\sigma$ contains
		a disjunction. There exists no single minimal model for the formula $P
		\vee Q$ because any model that satisfies both $P$ and $Q$ would not
		have a homomorphism to a model that satisfies the theory with only $P$
		or $Q$ in its set of facts. However, the set containing a model
		$\mathbb{M}$ that contains $P$ in its set of facts and a model
		$\mathbb{N}$ that contains $Q$ in its set of fact would be jointly
		minimal.

	\subsection{Positive Existential Form}

		Formul{\ae} in \emph{positive existential form} are constructed using
		only conjunctions ($\wedge$), disjunctions ($\vee$), existential
		quantifications ($\exists$), tautologies ($\top$), contradictions
		($\bot$), equalities, and relations.

		\newtheorem{pef-theorems}{Theorem}

		\begin{pef-theorems}
			The set of models of a sentence $\sigma$ is closed under homomorphisms
			if and only if $\sigma$ is logically equivalent to a positive
			existential formula $\varphi$.
		\end{pef-theorems}

	\subsection{Geometric Logic}
	\label{sec:technical_background.geometric_logic}

		\emph{Geometric logic} formul{\ae} are implicitly universally
		quantified implications between positive existential formul{\ae}. More
		specifically, a geometric logic formula is of the form
		\[\forall\ (free(F_L) \cup free(F_R)) : F_L \to F_R\]
		where $free$ is the function that returns the set of all free variables
		for a given formula and both $F_L$ and $F_R$ are are first-order logic
		formul{\ae} in positive existential form.

		A set of geometric logic formul{\ae} is called a \emph{geometric
		theory}.

		It is convention to treat a positive existential formula $\sigma$ as
		$\top \to \sigma$ when expecting a geometric logic formula. It is also
		convention to treat a negated positive existential formula $\neg\sigma$
		as $\sigma \to \bot$.

		Examples of geometric logic formul{\ae}:

		\begin{tabular}{lll}
		\emph{reflexivity}   & \qquad  &  $\top \to R(x,x)$                  \\
		\emph{symmetry}      & \qquad  &  $R(x,y) \to R(y,x)$                \\
		\emph{transitivity}  & \qquad  &  $R(x,y) \wedge R(y,z) \to R(x,z)$  \\
		\end{tabular}

		Negation of a relation $R$ with arity $k$ can be implemented by
		introducing another relation $R'$ with arity $k$, adding two
		formul{\ae} of the form $R \wedge R' \to \bot$ and $\top \to R \vee
		R'$, and using $R'$ where $\neg R$ would be used.


\section{The Chase}

	The \emph{chase} is a function that, when given a geometric theory, will
	generate a set of jointly minimal models for that theory. More
	specifically, if $\mathcal{U}$ is the set of all models obtained from an
	execution of the chase over a geometric theory $T$, for any model
	$\mathbb{M}$ such that $\mathbb{M} \models T$, there is a homomorphism from
	some model $\mathbb{U} \in \mathcal{U}$ to $\mathbb{M}$.

	Geometric logic formul{\ae} are used by the chase because they have the
	useful property where adding any relations or domain members to a model
	that satisfies a geometric logic formula will never cause the formula to no
	longer be satisfied. This is particularly helpful when trying to create a
	model that satisfies all formul{\ae} in a geometric theory.

	There are three types of runs of the chase:
	\begin{itemize}
	\item a set of jointly minimal models is found in finite time
	\item an empty result is found in finite time
	\item an infinite run with possible return dependent on implementation
	\end{itemize}

	\subsection{Algorithm}

		The chase starts with an input theory $T$ and a model $\mathbb{M}$ that
		has an empty domain and an empty set of facts.

		\begin{algorithm}[H]
		\DontPrintSemicolon
		\TitleOfAlgo{The Chase}
		\While{$\exists$ a formula $\sigma \in T$ such that $\mathbb{M} \not\models \sigma$}{
			choose formula $\sigma \in T$ such that $\mathbb{M} \not\models \sigma$ \;
			split this formula into the left and right sides of its implication, $\alpha$ and $\beta$ \;
			\If{$\beta$ is a contradiction}{halt with a failure \;}
			a binding $\lambda$ is chosen such that $\mathbb{M} \models_\lambda \alpha$ \;
			\If{$\beta$ contains a disjunction}{choose a disjunct and assign it to $\beta$ \;}
			\ForEach{free variable not in the domain of $\lambda$}{add a new domain member to the domain \;}
			\ForEach{new domain element $e$ and the variable $v$ for which it was added}{redefine $\lambda$ as $\lambda_{v \mapsto e}$ \;}
			\ForEach{atomic $R$}{replace each variable $v$ with $\lambda(v)$ and add it to the model \;}
		}
		halt with result $\mathbb{M}$ \;
		\end{algorithm}

	\subsection{Examples}

		Define $\Sigma$ as the following geometric theory.

		\begin{eqnarray}
			\label{eqn:chase1}
			\top    &  \to  &  \exists\ x,y : R(x,y)                             \\
			\label{eqn:chase2}
			R(x,y)  &  \to  &  (\exists\ z : Q(x,z)) \vee P                      \\
			\label{eqn:chase3}
			Q(x,y)  &  \to  &  (\exists\ z : R(x,z)) \vee (\exists\ z : R(z,y))  \\
			\label{eqn:chase4}
			P       &  \to  &  \bot
		\end{eqnarray}

		The following three chase runs show the different types of results
		depending on which disjunct the algorithm attempts to satisfy when a
		disjunction is encountered.

		\begin{enumerate}
		\item A non-empty result in finite time:

			\begin{tabular}{lllllll}
				$\emptyset$ & $\mapsto$ & \{ & $a,b$   & $|$ & $R(a,b)$         & \} \\
				{}          & $\mapsto$ & \{ & $a,b,c$ & $|$ & $R(a,b), Q(a,c)$ & \}
			\end{tabular}

			Since the left side of \eqref{eqn:chase1} is always satisfied, but its
			right side is not, domain members $a$ and $b$ and fact $R(a,b)$ are
			added to the initially empty model to satisfy \eqref{eqn:chase1}. The
			left side of \eqref{eqn:chase2} holds, but the right side does not, so
			one of the disjuncts $\exists\ z : Q(x,z)$ or $P(x)$ is chosen to
			be satisfied. Assuming the left operand is chosen, $x$ will already
			have been assigned to $a$ and a new domain member $c$ and a new
			fact $Q(a,c)$ will be added to satisfy \eqref{eqn:chase2}. With the
			current model, all rules hold under any environment. Therefore,
			this model is minimal.

		\item An empty result in finite time:

			\begin{tabular}{lllllll}
				$\emptyset$ & $\mapsto$ & \{ & $a,b$   & $|$ & $R(a,b)$               & \} \\
				{}          & $\mapsto$ & \{ & $a,b,c$ & $|$ & $R(a,b), P(a,c)$       & \} \\
				{}          & $\mapsto$ & \{ & $a,b,c$ & $|$ & $R(a,b), P(a,c), \bot$ & \} \\
				{}          & $\mapsto$ & \multicolumn{5}{l}{ $\varepsilon$ }
			\end{tabular}

			Again, domain members $a$ and $c$ and fact $R(a,b)$ are added to
			the initial model to satisfy \eqref{eqn:chase1}. This time, when
			attempting to satisfy \eqref{eqn:chase2}, the right side is chosen and
			$P$ is added to the set of facts. After adding this new fact, rule
			\eqref{eqn:chase4} no longer holds; its left side is satisfied, but its
			right side does not hold for all of the bindings for which it is
			satisfied. When we attempt to satisfy the right side of
			\eqref{eqn:chase4}, it is found to be a contradiction and therefore
			unsatisfiable. Since this model can never satisfy this theory, the
			chase fails.

		\item An infinite run:

			\begin{tabular}{lllllll}
				$\emptyset$ & $\mapsto$ & \{ & $a,b$        & $|$ & $R(a,b)$                                 & \} \\
				{}          & $\mapsto$ & \{ & $a,b,c$      & $|$ & $R(a,b), Q(a,c)$                         & \} \\
				{}          & $\mapsto$ & \{ & $a,\ldots,d$ & $|$ & $R(a,b), Q(a,c), R(d,c)$                 & \} \\
				{}          & $\mapsto$ & \{ & $a,\ldots,e$ & $|$ & $R(a,b), Q(a,c), R(d,c), Q(d,e)$         & \} \\
				{}          & $\mapsto$ & \{ & $a,\ldots,f$ & $|$ & $R(a,b), Q(a,c), R(d,c), Q(d,e), R(f,e)$ & \} \\
				{}          & $\mapsto$ & \multicolumn{5}{l}{ $\ldots$ }
			\end{tabular}

			Like in the example above that returned a non-empty, finite result,
			the first two steps add domain members $a$, $b$, and $c$ and facts
			$R(a,b)$ and $Q(a,c)$. The left side of the implication in
			\eqref{eqn:chase3} now holds, but the right side does not. In order to
			make the right side hold, one of the disjuncts needs to be
			satisfied. If the right disjunct is chosen, a new domain member $d$
			and a new relation $R(d,c)$ will be added. This will cause the
			left side of the implication in \eqref{eqn:chase2} to hold for $R(d,c)$,
			but the right side will not hold for the same binding. $Q(d,e)$ will
			be added, and this loop will continue indefinitely unless a
			different disjunct is chosen in \eqref{eqn:chase2} or \eqref{eqn:chase3}.

		\end{enumerate}

	\subsection{History}

		

	\subsection{Joint Minimality Theorems}

		\newtheorem{minimality-theorems}{Theorem}

		\begin{theorem}
			A geometric theory $T$ is satisfiable if and only if there is an
			infinite run, given disjunct and binding choice is fair, or there
			is a chase run of $T$ that returns a non-empty result.
		\end{theorem}

		\begin{proof}
			
		\end{proof}

		\begin{theorem}
			Let $\mathcal{M}$ be the set of models returned from all successful
			runs of the chase over a geometric theory $T$. For any model
			$\mathbb{N}$ such that $\mathbb{N} \models T$, there is a model
			$\mathbb{M} \in \mathcal{M}$ such that $\mathbb{M} \preceq
			\mathbb{N}$. $\mathcal{M}$ is jointly minimal.
		\end{theorem}

		\begin{proof}
			
		\end{proof}


\section{An Extended Application: \\ Cryptographic Protocol Analysis}

	The chase can be used for protocol analysis. A technique for the analysis
	of protocols has been defined wherein the essentially different runs of a
	protocol need to be known. These essentially different protocol runs are
	analagous to minimal models.  When a protocol is described using
	first-order logic, the chase can find such minimal models.

	The protocol can then be analysed for any characteristics such as the
	existence of security violations or other unexpected behaviour.

	\subsection{Background}

		\subsubsection{Strand Space}

			Strand space was developed as a method of formally reasoning about
			cryptographic protocols.  Participants in a protocol run are
			represented by strands. A single physical entity can be
			respresented as multiple strands if they play many roles in the
			protocol. A strand is made up of a sequence of nodes where every
			node either sends or receives a message.

			Adversary strands represent the actions of a participant that
			attempts to use the protocol for a purpose other than the one for
			which it was originally intended. Adversary strands are not bound
			by the rules defined by the protocol; they manipulate messages
			being sent and received by non-adversarial strands.

			The capabilities of the adversary strands are given by the
			Dolev-Yao Threat Model \textbf{references}. Adversary Strands are
			able to perform precisely five operations:

			\begin{enumerate}
			\item Pairing: The pairing of two terms
			\item Unpairing: The extraction of a term from a pair
			\item Encryption: Given a key $k$ and a plaintext $m$, the construction of the ciphertext $\{|m|\}_k$ by encrypting $m$ with $k$
			\item Decryption: Given a ciphertext $\{|m|\}_k$ and its decryption key $k^{-1}$, the extraction of the plaintext $m$ by decryping $\{|m|\}_k$ with $k^{-1}$.
			\item Generation: The generation of an original term; not assumed to be secure
			\end{enumerate}

			Pairing, encryption, and generation are construction operations.
			Decryption and unpairing are deconstruction operations.

		\subsubsection{Cremer's Algorithm}

			Cremer's Algorithm adds constraints to strand spaces to allow one
			to infer the possible shapes of protocol runs. A major constraint
			added by Cremer forces messages constructed by an adversary to be
			traced back to messages sent by non-adversarial strands.

			Two other constraints are normalisation and efficiency. A protocol
			is efficient an adversary always takes a messages from the earliest
			point at which they appear. A protocol is normal when the adversary
			always performs zero or more deconstruction operations followed by
			zero or more construction operations, except in the case of
			constructing decryption keys.

			Cremer proved that these constraints do not limit the capabilties
			of an adversary \textbf{reference}.

		\subsubsection{CPSA Algorithm}

			\textbf{ don't know yet }

	\subsection{The Problem}

		Protocol reasearchers want to be able to programmatically reason about
		cryptographic protocols. A common technique for this is to find a set of
		essentially different clsses of protocol runs. This can be accomplished by
		finding minimal models of a geometric logic representation of the protocol.
		This is precisely the problem the chase solves.

	\subsection{The Solution: Minimal Models}

		The minimal models $\mathcal{M}$ that the chase outputs are
		representative of all models because there always exists a homomorphism
		from some model $\mathbb{M} \in \mathcal{M}$ to any model that
		satisfies the theory. Models that satisfy the theory given to the chase
		in this use case represent classes of runs of the protocol. In this
		way, a set of jointly minimal models represents every possible run of
		the protocol. Finding every possible run of the protocol is prohibitive
		because there are infinitely many. Because the set of models is
		countably infinite, they can be enumerated, but can not be listed.

	\subsection{Designing An Analagous Theory}

		In order to create a geometric theory describing a protocol, the
		formul{\ae} that define strand spaces, normilisation, efficiency, and
		Cremer's algorithm must be derived. The formul{\ae} defining the
		protocol must be combined with the scaffolding to create a theory that
		can be used to infer the possible runs of the protocol.

		The half-duplex protocol was chosen to be used as an example.  This
		protocol involves two participants, Alice and Bob. Alice sends Bob a
		nonce that she generated, encrypted with Bob's public key. Bob receives
		the nonce encrypted with his public key. Bob replies to Alice with the
		decypted nonce. Alice receives Bob's message.

		These rules were generated manually by direct translation into
		geometric logic. Ideally, the process of generating geometric logic
		formul{\ae} from protocols should be done automatically.

	\subsection{The Results}

		The chase was run on the logic representation of the half-duplex protocol.
		A single model was returned during the execution of the algorithm,
		which was manually stopped before natural completion. This model, like
		all models returned by the chase, satisfies the input theory and
		belongs to a set of jointly minimal models for the theory.

\section{Haskell Chase Implementation}

	The goal of the implementation of the chase is to deterministically find
	all possible outcomes of the chase. It does this by forking and taking all
	paths when encountering a disjunct rather than nondeterministically
	choosing one disjunct to satisfy.

	The results from the attempts to satisfy each disjunct are returned as a
	list. The returned list will not contain an entry for runs that return no
	model, and will merge lists returned from runs that themselves encountered
	a disjunct. The lazy evaluation of Haskell allows a user to access members
	of the returned list even though some chase runs have not returned a value.

	Appendix B contains the chase-running portions of the implementation.

	\subsection{Operation}

	The first step of the chase implementation is to verify that each formula
	of the given theory is a geometric logic formula. If a formula $\varphi$ is
	not a geometric logic formula, the chase tries to coerce it into one using
	the following algorithm:

	\begin{algorithm}[H]
	\DontPrintSemicolon
	\Switch{$\varphi$}{
		\Case{$\neg\alpha$}{
			\If{$\alpha$ is in positive existential form}{
				$\varphi$ is replaced with $\alpha \to \bot$ \;
			}
			\lElse{error \;}
		}
		\Other{
			\If{$\varphi$ is in positive existential form}{
				$\varphi$ is replaced with $\top \to \varphi$ \;
			}
			\lElse{error \;}
		}
	}
	\end{algorithm}

	After the input verification and coercion step, the $chase$ function sorts
	the input formul{\ae} by the number of disjuncts on the right side of the
	implication. This step will cause each fork of the algorithm to finish in
	less time, as they are likely to halt before forking yet again.

	Once the input formul{\ae} are sorted, the $chase$ function begins
	processing a \emph{pending} list, which is initially populated with a
	single model that has an empty domain and no facts. In the special case
	where $chase$ is run on an empty list, an empty list of models is returned.

	For each \emph{pending} model, each formula is evaluated to see if it holds
	in the model for all environments.  If an envionment is found that does not
	satisfy the model, the model and environment in which the formula did not
	hold is passed to the $satisfy$ function, along with the formula that needs
	to be satisfied.  The list of models returned from $satisfy$ is merged into
	the \emph{pending} list, and the result of running $chase$ on the new
	\emph{pending} list is returned.  If, however, the model holds for all
	formul{\ae} in the theory and all possible associated environments, it is
	concatenated with the result of running the chase on the rest of the models
	in the \emph{pending} list.

	The $satisfy$ function performs a pattern match on the type of formula
	given. Assuming $satisfy$ is given a model $\mathbb{M}$, an environment
	$\lambda$, and a formula $\varphi$, $satisfy$ will behave as outlined
	in the fllowing algorithm.

	\begin{algorithm}[H]
	\DontPrintSemicolon
	\TitleOfAlgo{satisfy :: Model $\to$ Environment $\to$ Formula $\to$ [Model]}
	return \Switch{$\varphi$}{
		\lCase{$\top$}{return a list containing $\mathbb{M}$} \;
		\lCase{$\bot$}{return an empty list} \;
		\lCase{$x=y$}{return a list containing $quotient(\mathbb{M})$} \;
		\lCase{$\alpha \vee \beta$}{return $satisfy(\alpha) \cup satisfy(\beta)$} \;
		\Case{$\alpha \wedge \beta$}{
			create an empty list $r$ \;
			\ForEach{model $m$ in $satisfy(\alpha)$}{
				union $r$ with $satisfy(\beta)$ \;
			}
			return $r$ \;
		}
		\Case{$\alpha \to \beta$}{
			\lIf{$\mathbb{M}_\lambda \models \alpha$}{return $satisfy(\beta)$ \;}
			\lElse{return an empty list \;}
		}
		\Case{$R[\vec x]$}{
			define a new model $\mathbb{N}$ where $|\mathbb{N}| = |\mathbb{M}|$ \;
			add a new element $\omega$ to $|\mathbb{N}|$ \;
			\lForAll{$P_\mathbb{M}$}{$P_\mathbb{N} = P_\mathbb{M}$ \;}
			define $R_{\mathbb{N}}[x_0 \ldots x_n]$ as $R_{\mathbb{M}}[\lambda(x_0) \ldots \lambda(x_n)]$ \;
			\ForEach{$v \in \vec x$}{
				\lIf{$v \not\in \lambda$}{
					$\lambda$ becomes $\lambda_{v \mapsto \omega}$ \;
				}
			}
			return a list containing $\mathbb{N}$ \;
		}
		\Case{$\exists\ \vec x : \alpha$}{
			\lIf{$\vec x = \emptyset$}{ recurse on $\alpha$ } \;
			\eIf{$|\mathbb{M}| \ne \emptyset$ and $\exists\ v' \in |\mathbb{M}| : (\lambda' = \lambda_{x_0 \mapsto v'}$ and $\mathbb{M} \models_{\lambda'} \alpha)$}{
				return a list containing $\mathbb{M}$ \;
			}{
				define a new model $\mathbb{N}$ where $|\mathbb{N}| = |\mathbb{M}|$ \;
				add an element $\omega$ to $|\mathbb{N}|$ such that $\omega \not\in |\mathbb{N}|$ \;
				\lForAll{$R_\mathbb{M}$}{$R_\mathbb{N} = R_\mathbb{M}$ \;}
				define $\kappa = \lambda_{x_0 \mapsto \omega}$ \;
				using model $\mathbb{N}$ and environment $\kappa$, return $satisfy(\exists\ \{x_1 \ldots x_n\} : \alpha)$ \;
			}
		}
	}
	\end{algorithm}

	\subsection{Input Format}

		Input to the parser must be in a form parsable by the following
		context-free grammar. Terminals are denoted by a {\tt monospace style}
		and nonterminals are denoted by an $oblique style$. The greek letter
		$\varepsilon$ matches a zero-length list of tokens. Patterns that match
		non-literal terminals are defined in the table following the grammar.

		\begin{tabular}{ll}
		$program$ & :    $\varepsilon$ \\
		{} & $\mid$      $exprList$ $optNEWLINE$ \\
		\\

		$exprList$ & :   $expr$ \\
		{} & $\mid$      $exprList$ \tt{NEWLINE} $expr$ \\
		\\

		$expr$ & :       \tt{TAUTOLOGY} \\
		{} & $\mid$      \tt{CONTRADICTION} \\
		{} & $\mid$      $expr$ \tt{OR} $expr$ \\
		{} & $\mid$      $expr$ \tt{AND} $expr$ \\
		{} & $\mid$      \tt{NOT} $expr$ \\
		{} & $\mid$      $expr$ \tt{->} $expr$ \\
		{} & $\mid$      \tt{->} $expr$ \\
		{} & $\mid$      $atomic$ \\
		{} & $\mid$      \tt{VARIABLE EQ VARIABLE} \\
		{} & $\mid$      \tt{FOR\_ALL} $argList$ $optCOLON$ $expr$ \\
		{} & $\mid$      \tt{THERE\_EXISTS} $argList$ $optCOLON$ $expr$ \\
		{} & $\mid$      \tt{(} $expr$ \tt{)} \\
		{} & $\mid$      \tt{[} $expr$ \tt{]} \\
		\\

		$atomic$ & :     \tt{PREDICATE} $index$ \\
		\\

		$index$ & :      \tt{(} $argList$ \tt{)} \\
		{} & $\mid$      \tt{[} $argList$ \tt{]} \\
		\\

		$argList$ & :    $arg$ \\
		{} & $\mid$      $argList$ \tt{,} $arg$ \\
		\\

		$arg$ & :        \tt{VARIABLE} \\
		\\

		$optCOLON$ & :   $\varepsilon$ \\
		{} & $\mid$      \tt{:} \\
		\\

		$optNEWLINE$ & : $\varepsilon$ \\
		{} & $\mid$      \tt{NEWLINE} \\
		\end{tabular}

		{\tt \begin{tabular}{|l|l|}
			\hline
			\textbf{Input Pattern} & \textbf{Terminal} \\
			\hline
			|                    & OR  \\
			\&                   & AND \\
			!                    & NOT \\
			=                    & EQ  \\
			$[$Tt$]$autology     & TAUTOLOGY \\
			$[$Cc$]$ontradiction & CONTRADICTION \\
			$[\backslash$r$\backslash$n$]$+ & NEWLINE \\
			$[$a-z$][$A-Za-z0-9\_'$]$* & VARIABLE \\
			$[$A-Z$][$A-Za-z0-9\_'$]$* & PREDICATE \\
			For$[$Aa$]$ll        & FOR\_ALL \\
			Exists               & THERE\_EXISTS \\
			\hline
		\end{tabular} }

		Comments are removed at the lexical analysis step and have no effect on
		the input to the parser.  Single-line comments begin with either a hash
		({\tt \#}) or double-dash ({\tt --}). Multiline comments begin with
		{\tt /*} and are terminated by {\tt */}.

	\subsection{Options}

		\subsubsection{I/O}

		When no options are given to the executable output by Haskell, it
		expects input from stdin and outputs models in a human-readable format
		to stdout. To take input from a file instead, pass the executable the
		{\tt -i} or {\tt --input} option followed by the filename.

		To output models to numbered files in a directory, pass the {\tt -o} or
		{\tt --output} option along with an optional directory name. The given
		directory does not have to exist. If the output directory is omitted,
		it defaults to ``{\tt ./models}".

		\subsubsection{Tracing}

		\textbf{not yet implemented}

	\subsection{Future Considerations}


\singlespacing

%\section{For later reference:}
%	\begin{figure}[graphics_test]
%		\begin{center}
%			\includegraphics[width=0.8\textwidth]{pic1.jpg}
%		\end{center}
%		\caption[Caption in List of Figures]{Caption in report}
%	\end{figure}


\appendix
	\section{Table of Syntax}
	\begin{tabular}{|r|l|}
		\hline
		\textbf{syntax}                  &  \textbf{definition}                                                     \\
		\hline
		$f^{-1}$                         &  the inverse function of $f$                                             \\
		\hline
		$R[a_0,a_1,a_2]$                 &  a \emph{relation} of: relation symbol $R$, arity $3$, and tuple $\langle a_0,a_1,a_2 \rangle$  \\
		$\top$                           &  a \emph{tautological formula}; one that will always hold                \\
		$\bot$                           &  a \emph{contradictory formula}; one that will never hold                \\
		$\rho = \tau$                    &  given assumed environment $\lambda$, $\lambda(\rho) = \lambda(\tau)$    \\
		$\neg \alpha$                    &  $\alpha$ does not hold                                                  \\
		$\alpha \wedge \beta$            &  both $\alpha$ and $\beta$ hold                                          \\
		$\alpha \vee \beta$              &  either $\alpha$ or $\beta$ hold                                         \\
		$\alpha \to \beta$               &  either $\alpha$ does not hold or $\beta$ holds                          \\
		$\forall x : \alpha$             &  for each member of the domain as $x$, $\alpha$ holds                    \\
		$\forall \vec{x} : \alpha$       &  for each $x_i \in x$, $\forall x_i : \alpha$ holds                      \\
		$\exists x : \alpha$             &  for at least one member of the domain as $x$, $\alpha$ holds            \\
		$\exists \vec{x} : \alpha$       &  for each $x_i \in x$, $\exists x_i : \alpha$ holds                      \\
		\hline
		$\lambda[x \mapsto y]$           &  the environment $\lambda$ with variable $x$ mapped to domain member $y$ \\
		\hline
		$\mathbb{M} \models_l \sigma$    &  $\mathbb{M}$ \emph{is a model of} $\sigma$ under environment $l$        \\
		$\mathbb{M} \models \sigma$      &  $\mathbb{M} \models_l \sigma$ given any environment $l$                 \\
		$\mathbb{M} \models_l \Sigma$    &  for each $\sigma \in \Sigma$, $\mathbb{M} \models_l \sigma$             \\
		$\mathbb{M} \models \Sigma$      &  for each $\sigma \in \Sigma$, $\mathbb{M} \models \sigma$               \\
		\hline
		$\Sigma \models \sigma$          &  $\Sigma$ \emph{entails} $\sigma$                                        \\
		\hline
		$\mathbb{M} \preceq \mathbb{N}$  &  there exists a \emph{homomorphism} $h : |\mathbb{M}| \to |\mathbb{N}|$  \\
		$\mathbb{M} \simeq \mathbb{N}$   &  $\mathbb{M}$ and $\mathbb{N}$ are \emph{homomorphically equivalent}     \\
		\hline
	\end{tabular}

	\section{Chase code}
	{\scriptsize
		\lstset{
			language=Haskell,
			numbers=left,
			columns=fixed,
			tabsize=3,
		}
		\lstinputlisting{../chase.hs}
	}



\addcontentsline{toc}{section}{References}
\begin{thebibliography}{9999}%\enlargethispage{\baselineskip}

\bibitem{AC}A~Cottrell,
	\textsl{Word Processors: Stupid and Inefficient},
	\\ \mbox{}\hfill\texttt{www.ecn.wfu.edu/\~{}cottrell/wp.html}

\end{thebibliography}



\end{document}
