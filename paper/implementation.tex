\section{Haskell Chase Implementation}
\label{sec:implementation}

	The goal of the implementation of the chase is to deterministically find
	all possible outcomes of the chase. It does this by forking and taking all
	paths when encountering a disjunct rather than nondeterministically
	choosing one disjunct to satisfy.

	The results from the attempts to satisfy each disjunct are returned as a
	list. The returned list will not contain an entry for runs that return no
	model, and will merge lists returned from runs that themselves encountered
	a disjunct. The lazy evaluation of Haskell allows a user to access members
	of the returned list as they are found, even though some chase runs have
	not returned a value, and even if a chase run is infinite.

	To be sure that a chase implementation returns every model a chase run
	could possibly return, it is important that the implementation is
	\emph{fair}. Recall the definition of fairness from section
	\ref{fairness_definition}.

	Though the discussed implementation is efficient, it is not fair. The
	domain is represented by an ordered type. Any time the algorithm would ask
	us to choose a binding, each variable is assigned a member of the domain,
	starting with all variables paired with the representative with the lowest
	ordering. Successive pairings relate variables with successive domain
	members. Problems similar to this are the cause of unfairness in the
	implementation.

	Appendix \ref{sec:appendix_chase} contains the chase-running portions of the
	implementation.

	\subsection{Operation}

		The first step of the chase implementation is to make sure that each
		formula of the given theory can be represented as a geometric logic
		formula. If a formula $\varphi$ can not be coerced to a geometric logic
		formula, the implementation tries to coerce it into one by applying the
		following rules recursively:

		\begin{align*}
		\neg\alpha \wedge \neg\beta         \quad & \mapsto \quad \alpha \vee \beta        \\
		\neg\alpha \vee \neg\beta           \quad & \mapsto \quad \alpha \wedge \beta      \\
		\neg\neg\alpha                      \quad & \mapsto \quad \alpha                   \\
		\neg\alpha \to \beta                \quad & \mapsto \quad \alpha \vee \beta        \\
		\alpha \to \neg\beta                \quad & \mapsto \quad \alpha \wedge \beta      \\
		\neg(\forall\ \vec{x} : \neg\alpha) \quad & \mapsto \quad \exists\ \vec{x} : \alpha
		\end{align*}

		All other constructs are preserved. If this transformed formula is still
		not in positive existential form, an error is thrown.

		The chase function then sorts the input formul{\ae} by the number of
		disjunctions on the right side of the implication. It allows branches
		to sometimes terminate where they could otherwise grow to an
		unnecessarily (and possibly infinitely) large size. This step will
		cause each branch of the algorithm to finish in less time, as they are
		likely to halt before branching yet again. The formul{\ae} are not
		sorted purely by absolute number of disjunctions on the right side, but
		by whether there are zero, one, or many disjunctions. This is done to
		avoid unnecessary re-ordering for no gain because formul{\ae} with no
		disjunctions or only a single disjunct on the right side are more
		likely to cause a branch to stop growing than one with many
		disjunctions. Likewise, formul{\ae} with zero disjunctions are more
		likely to cause a branch to halt than those with one or more
		disjunctions. A secondary sort also occurs within these classifications
		that orders formul{\ae} by the number of variables to reduce the number
		of bindings generated.

		Once the input formul{\ae} are sorted, the {\tt chase} function begins
		processing a \emph{pending} list, which is initially populated with a
		single model that has an empty domain and no facts.

		For each \emph{pending} model, each formula is evaluated to see if it
		holds in the model for all environments. If an environment is found
		that does not satisfy the model, the model and environment in which the
		formula did not hold is passed to the {\tt satisfy} function, along
		with the formula that needs to be satisfied. The list of models
		returned from {\tt satisfy} is merged into the \emph{pending} list, and
		the result of running {\tt chase} on the new \emph{pending} list is
		returned. If, however, the model holds for all formul{\ae} in the
		theory and all possible associated environments, it is concatenated
		with the result of running the chase on the rest of the models in the
		\emph{pending} list and returned.

		The {\tt satisfy} function performs a pattern match on the type of
		formula given. {\tt satisfy} will behave as outlined in the algorithm
		which can be found in Appendix \ref{sec:appendix_satisfy}.

	\subsection{Input Format}

		Input to the program must be in a form parsable by the context-free
		grammar seen in Appendix \ref{sec:appendix_parser}. Terminals are
		denoted by a {\tt monospace style} and nonterminals are denoted by an
		$oblique\ style$. The Greek letter $\varepsilon$ matches a zero-length
		list of tokens. Patterns that match non-literal terminals are defined
		in the table in Appendix \ref{sec:appendix_lexer}. The expected input
		is essentially a newline-separated list of ASCII representations of
		geometric formul{\ae}.

		Comments are removed at the lexical analysis step and have no effect on
		the input to the parser. Single-line comments begin with either a hash
		({\tt \#}) or double-dash ({\tt --}). Multi-line comments begin with
		{\tt /*} and are terminated by {\tt */}.

	\subsection{Options}

		Help on the usage of the chase implementation can be found by passing
		the executable output by Haskell the {\tt --help} or {\tt -?} options.

		When no options are given to the executable, it expects input from
		stdin and outputs models in a human-readable format to stdout. To take
		input from a file instead, pass the executable the {\tt -i} or {\tt
		--input} option followed by the filename.

		To output models to numbered files in a directory, pass the {\tt -o} or
		{\tt --output} option along with an optional directory name. The given
		directory does not have to exist. If the output directory is omitted,
		it defaults to ``{\tt ./models}".

		Using the {\tt -o} or {\tt --output} options will change the selection
		for output format to a machine-readable format. To switch output formats
		at any time, pass the {\tt -h} or {\tt -m} flags for human-readable and
		machine-readable formats respectively.


	\subsection{Future Considerations}

		This section details areas of possible improvement/development.

		\subsubsection{Better Data Structures}

			Some less-than-optimal data structures are being used to hold data
			that should really be in a Data.Map or Data.Set. One such example
			of this is with the truth table holding the relation information of
			a model. This truth table should be implemented as a Data.Map.
			Environments are currently a list of tuples, but should really be a
			Data.Map. Instead of Domains being a list of DomainMember, it
			would be better if a Domain was a Data.Set.

		\subsubsection{Broader Use of the Maybe Monad}

			In several helper functions, the program's execution is halted and
			an error is output when the function receives certain invalid
			inputs. These functions should take advantage of the Maybe Monad
			and return {\tt Maybe a} where {\tt a} is the type they currently
			return. One particular example of this is the {\tt pef} function.
			When {\tt pef} takes a formula as input that can not be converted
			to positive-existential form, it causes the program to produce an
			error and exit. Instead, {\tt pef} should return {\tt Maybe
			Formula} and the places where it is used should handle the error
			condition however they choose.

		\subsubsection{Binding Search Approach for Satisfaction Checking}

			When checking if a model with 30 domain members and 60 facts holds
			for a formula with 5 universally quantified variables (a reasonable
			real-life usage example), $30^5$ or $24,300,000$ bindings will have
			to be generated, and the formula will be checked under each one.
			But by limiting the checked bindings to only those that can produce
			facts that exist in the model from the atomics in the formula, only
			the 60 facts will have to be checked in the worst case, and only 1
			in the best. In practical use, this should dramatically reduce the
			running time of theories that produce finite results.

		\subsubsection{Avoid Isomorphic Model Generation}

			By taking different paths to arrive at the same model, the chase
			often creates equivalent models that satisfy its given theory.
			These duplicate models are already being filtered from the output.
			However, another problem exists when the chase returns isomorphic
			models. Given two models, it is very computationally expensive to
			determine if there exists an isomorphism between them. If a fast
			method of determining if two models are isomorphic is found,
			including an implementation of it would provide more valuable
			results to the user.

		\subsubsection{Improve Efficiency}

			The main goals of this project were to write a correct chase
			implementation and apply it in a real-world situation. While
			reasonable and obvious optimizations were made, there is still
			plenty of room for optimization.

			In \cite{Harrison09} Harrison's \emph{Practical Logic and Automated
			Reasoning}, Harrison mentions a large number of formula rewriting
			and simplification algorithms, many of which are already
			implemented in the Helpers module. Those functions that are
			written, however, are not currently being used by the chase
			functions, and there are surely other functions that Harrison
			mentions that have not yet been written.

			One specific example of a function that \cite{Harrison09} Harrison
			mentions on pages 141 to 144 is {\tt pullQuants}, which pulls all
			quantifiers in a formula to the outside. This results in a formula
			with no conjunctions or disjunctions of quantified subformul{\ae}.
			This function is implemented, but a function that does exactly the
			opposite of this will help speed up satisfaction checking when
			using the traditional looping method. This will minimize the number
			of times bindings need to be generated for all permutations of the
			current model's domain because the number of quantified variables
			will be reduced.

		\subsubsection{Parallelize}

			The chase implementation should be easily parallelizable. The
			program can fork for every call to the {\tt branch} function as it
			is mapped over a pending list in the {\tt chase'} function. As each
			fork finds a contradiction, the forks will die. This method could
			still lead to a large number of threads. In a proper implementation,
			forks that are waiting on a single child could consume that child's
			work to minimize this problem. Doing this would prevent long chains
			of waiting threads.

		\subsubsection{Tracing}

			Currently, Debug.Trace is being used to output real-time status for
			ease of debugging. This output is helpful to both developers of
			the chase implementation and theories that will be given as input.
			Unfortunately, all of the output can really slow down a chase run
			of a simple theory. Currently, the only way to disable tracing is
			to replace the definition
				\[
				trace = Debug.Trace.trace
				\]
			with
				\[
				trace\ x = id
				\]

			A flag is already being read in from the command line as {\tt -d}
			and stored in the options record under {\tt optDebug}. When this
			flag is found, a callback function is being invoked in the Main
			module. Someone who implements this enhancement would need to be
			able to alter the behaviour of the Chase module's $trace$ function
			from a function within the Main module. Ideally, a better
			debugging output method will be found and Debug.Trace.trace will no
			longer need to be used in what is otherwise production-ready code.
