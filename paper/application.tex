\section{An Extended Application: \\ Cryptographic Protocol Analysis}

	The chase can be used for protocol analysis. A common technique for the analysis
	of protocols involves identifying the \emph{essentially different} runs of
	the protocol. These essentially different protocol runs are analogous to
	minimal models. When a protocol is described using geometric logic, the
	chase can find such minimal models.

	The protocol can then be analysed for characteristics such as the
	existence of security violations or other unexpected behaviour.

	\subsection{Background}

		\subsubsection{Strand Spaces}

			The \emph{strand space formalism} was developed as a method of
			formally reasoning about cryptographic protocols. Participants in a
			protocol run are represented by \emph{strands}. A single physical
			entity can be represented as multiple strands if they play many
			roles in the protocol. A strand is made up of a sequence of nodes
			where every node either sends or receives a message.

			\emph{Adversary strands} represent the actions of participants
			that attempt to use the protocol for a purpose other than the one
			for which it was originally intended. Adversary strands are not
			bound by the rules defined by the protocol; they manipulate
			messages being sent and received by non-adversarial strands.

			The capabilities of the adversary strands are given by the
			Dolev-Yao Threat Model \textbf{references}. Adversary strands are
			able to perform precisely five operations:

			\begin{description}
			\item [pairing] the pairing of two terms
			\item [unpairing] the extraction of a term from a pair
			\item [encryption] given a key $k$ and a plaintext $m$, the construction of the ciphertext $\{|m|\}_k$ by encrypting $m$ with $k$
			\item [decryption] given a ciphertext $\{|m|\}_k$ and its decryption key $k^{-1}$, the extraction of the plaintext $m$
			\item [generation] the generation of an original term, which is not assumed to be secure
			\end{description}

			Pairing, encryption, and generation are construction operations.
			Decryption and unpairing are deconstruction operations.

		\subsubsection{Cremers' Algorithm}

			\emph{Cremers' Algorithm} adds constraints to adversarial actions
			to allow one to infer the possible shapes of protocol runs.

			Cremers gives two major constraints. A protocol is \emph{efficient}
			if an adversary always takes a message from the earliest point at
			which it appears. A protocol is \emph{normal} when the adversary
			always performs zero or more deconstruction operations followed by
			zero or more construction operations, except in the case of
			constructing decryption keys.

			Cremers proved that these constraints do not limit the capabilities
			of an adversary \textbf{reference}.

			An important insight used in Cremers' algorithm, called
			\emph{chaining}, states that terms in messages received from an
			adversary strand always originate in a non-adversarial strand.

	\subsection{The Problem}

		Protocol researchers want to be able to programmatically reason about
		cryptographic protocols. A common technique for this is to find a set of
		essentially different classes of protocol runs. This can be accomplished by
		finding minimal models of a geometric logic representation of the protocol.
		This happens to be precisely the problem the chase solves.

	\subsection{The Solution: Minimal Models}

		Given a theory $\mathcal{T}$, the jointly minimal models $\mathcal{M}$
		that the chase outputs are representative of all models because there
		always exists a homomorphism from some model $\mathbb{M} \in
		\mathcal{M}$ to any model that satisfies $\mathcal{T}$. Each model that
		satisfies $\mathcal{T}$ represents a class of runs of the protocol. The
		set of all models output by the chase represents every possible run of
		the protocol. Finding every possible run of the protocol is prohibitive
		because there are infinitely many. Because the set of models is
		countably infinite, they can be enumerated, but can not be listed.

	\subsection{Designing An Analogous Theory}

		In order to create a geometric theory describing a protocol, the
		formul{\ae} that define strand spaces, normilisation, efficiency, and
		chaining must be derived. The formul{\ae} defining the protocol must be
		combined with this scaffolding to create a theory that can be used to
		infer the possible runs of the protocol.

		The \emph{half-duplex protocol} was chosen to be used as an example.
		This protocol involves two participants, Alice and Bob. The protocol
		specifies that the following actions take place:

		\begin{enumerate}
		\item Alice sends Bob a nonce that she generated, encrypted with Bob's public key
		\item Bob receives the encrypted nonce
		\item Bob replies to Alice with the decrypted nonce
		\item Alice receives Bob's message
		\end{enumerate}

		The geometric logic rules that model this protocol were generated
		manually by direct translation into geometric logic. Ideally, the
		process of generating geometric logic formul{\ae} from protocols should
		be done automatically.

	\subsection{The Results}

		The chase was run on the logic representation of the half-duplex protocol.
		A single model was returned during the execution of the algorithm,
		which was manually stopped before natural completion. This model, like
		all models returned by the chase, satisfies the input theory, and
		belongs to a set of jointly minimal models for the theory. The returned
		model denotes a run of the protocol which contains no adversary strands
		and is a correct execution of the protocol.
