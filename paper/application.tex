\section{An Extended Application: \\ Cryptographic Protocol Analysis}

	The chase can be used for protocol analysis. A technique for the analysis
	of protocols has been defined wherein the essentially different runs of a
	protocol need to be known. These essentially different protocol runs are
	analagous to minimal models.  When a protocol is described using
	first-order logic, the chase can find such minimal models.

	The protocol can then be analysed for any characteristics such as the
	existence of security violations or other unexpected behaviour.

	\subsection{Background}

		\subsubsection{Strand Space}

			Strand space was developed as a method of formally reasoning about
			cryptographic protocols.  Participants in a protocol run are
			represented by strands. A single physical entity can be
			respresented as multiple strands if they play many roles in the
			protocol. A strand is made up of a sequence of nodes where every
			node either sends or receives a message.

			Adversary strands represent the actions of a participant that
			attempts to use the protocol for a purpose other than the one for
			which it was originally intended. Adversary strands are not bound
			by the rules defined by the protocol; they manipulate messages
			being sent and received by non-adversarial strands.

			The capabilities of the adversary strands are given by the
			Dolev-Yao Threat Model \textbf{references}. Adversary Strands are
			able to perform precisely five operations:

			\begin{enumerate}
			\item Pairing: The pairing of two terms
			\item Unpairing: The extraction of a term from a pair
			\item Encryption: Given a key $k$ and a plaintext $m$, the construction of the ciphertext $\{|m|\}_k$ by encrypting $m$ with $k$
			\item Decryption: Given a ciphertext $\{|m|\}_k$ and its decryption key $k^{-1}$, the extraction of the plaintext $m$ by decryping $\{|m|\}_k$ with $k^{-1}$.
			\item Generation: The generation of an original term; not assumed to be secure
			\end{enumerate}

			Pairing, encryption, and generation are construction operations.
			Decryption and unpairing are deconstruction operations.

		\subsubsection{Cremer's Algorithm}

			Cremer's Algorithm adds constraints to strand spaces to allow one
			to infer the possible shapes of protocol runs. A major constraint
			added by Cremer forces messages constructed by an adversary to be
			traced back to messages sent by non-adversarial strands.

			Two other constraints are normalisation and efficiency. A protocol
			is efficient an adversary always takes a messages from the earliest
			point at which they appear. A protocol is normal when the adversary
			always performs zero or more deconstruction operations followed by
			zero or more construction operations, except in the case of
			constructing decryption keys.

			Cremer proved that these constraints do not limit the capabilties
			of an adversary \textbf{reference}.

		\subsubsection{CPSA Algorithm}

			\textbf{ don't know yet }

	\subsection{The Problem}

		Protocol reasearchers want to be able to programmatically reason about
		cryptographic protocols. A common technique for this is to find a set of
		essentially different clsses of protocol runs. This can be accomplished by
		finding minimal models of a geometric logic representation of the protocol.
		This is precisely the problem the chase solves.

	\subsection{The Solution: Minimal Models}

		The minimal models $\mathcal{M}$ that the chase outputs are
		representative of all models because there always exists a homomorphism
		from some model $\mathbb{M} \in \mathcal{M}$ to any model that
		satisfies the theory. Models that satisfy the theory given to the chase
		in this use case represent classes of runs of the protocol. In this
		way, a set of jointly minimal models represents every possible run of
		the protocol. Finding every possible run of the protocol is prohibitive
		because there are infinitely many. Because the set of models is
		countably infinite, they can be enumerated, but can not be listed.

	\subsection{Designing An Analagous Theory}

		In order to create a geometric theory describing a protocol, the
		formul{\ae} that define strand spaces, normilisation, efficiency, and
		Cremer's algorithm must be derived. The formul{\ae} defining the
		protocol must be combined with the scaffolding to create a theory that
		can be used to infer the possible runs of the protocol.

		The half-duplex protocol was chosen to be used as an example.  This
		protocol involves two participants, Alice and Bob. Alice sends Bob a
		nonce that she generated, encrypted with Bob's public key. Bob receives
		the nonce encrypted with his public key. Bob replies to Alice with the
		decypted nonce. Alice receives Bob's message.

		These rules were generated manually by direct translation into
		geometric logic. Ideally, the process of generating geometric logic
		formul{\ae} from protocols should be done automatically.

	\subsection{The Results}

		The chase was run on the logic representation of the half-duplex protocol.
		A single model was returned during the execution of the algorithm,
		which was manually stopped before natural completion. This model, like
		all models returned by the chase, satisfies the input theory and
		belongs to a set of jointly minimal models for the theory.
