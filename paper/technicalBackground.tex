\section{Technical Background}

	\subsection{Definitions}

		In this paper, logic symbols and other possibly ambiguous or uncommon
		notation will be used extensively, and thusly must be clearly defined.

		\subsubsection{Models}

			A \emph{model} $\mathbb{M}$ is a construct that consists of:
			\begin{itemize}
			\item a set, referenced as $|\mathbb{M}|$, called the \emph{universe} or \emph{domain} of $\mathbb{M}$
			\item a set of pairings of a \emph{predicate} and a non-negative integral arity
			\item for each predicate $R$ with arity $k$, a relation $R^\mathbb{M}_k \subseteq |\mathbb{M}|$
			\end{itemize}
			It is important to distinguish the predicate, which is just a symbol,
			from the relation that it refers to when paired with its arity. The
			relation itself is a set of tuples of elements from the universe.

		\subsubsection{First-order Logic}

			\emph{First-order logic}, also called \emph{predicate logic}, is a
			formula logic system that is an extension of propositional logic. For
			our purposes, this logic system will not contain any constant symbols
			or function symbols, which are commonly included in first-order and
			propositional logic.

			A first-order logic formula is defined inductively by
			\begin{itemize}
			\item if $R$ is a relation symbol of arity $k$ and each of $t_0 , \ldots , t_{k-1}$ is a variable, then $R[t_0,\ldots,t_{k-1}]$ is a formula, specifically an \emph{atomic formula}
			\item if $\alpha$ is a formula then $(\neg\alpha)$ is a formula
			\item if $\alpha$ and $\beta$ are formulas then $(\alpha\wedge\beta)$ is a formula
			\item if $\alpha$ and $\beta$ are formulas then $(\alpha\vee\beta)$ is a formula
			\item if $\alpha$ and $\beta$ are formulas then $(\alpha\to\beta)$ is a formula
			\item if $\alpha$ is a formula and $x$ is a variable then $(\forall x : \alpha)$ is a formula
			\item if $\alpha$ is a formula and $\vec{x}$ is a set of variables of size $k$ then $(\forall \vec{x} : \alpha)$ is $(\forall x_0 \ldots \forall x_{k-1} : \alpha)$
			\item if $\alpha$ is a formula and $x$ is a variable then $(\exists x : \alpha)$ is a formula
			\item if $\alpha$ is a formula and $\vec{x}$ is a set of variables of size $k$ then $(\exists \vec{x} : \alpha)$ is $(\forall x_0 \ldots \forall x_{k-1} : \alpha)$
			\end{itemize}

		\subsubsection{Geometric Logic}

			\emph{Geometric logic} is first-order logic with constraints on the
			shape of the expression.  Geometric logic formulas are implicitly
			universally quantified first-order logic expressions of the form \[A_0
			\wedge \ldots \wedge A_n \to E_0 \vee \ldots \vee E_m\] where $A_0
			\ldots A_n$ are atomics, $E_0 \ldots E_m$ are first-order logic
			expressions of the form $\exists_{x_0 \ldots x_k} A_0 \wedge \ldots
			\wedge \exists_{x_0 \ldots x_p} A_y$, and $n$, $m$, $k$, $p$, and $y$
			are integers greater than or equal to $0$. A set of geometric logic
			formulas is called a \emph{geometric theory}.

			\textbf{explain why GL is useful to us}

		\subsubsection{Variable Binding, Environments}

			A \emph{sentence} is a formula with no free variables.

			An \emph{environment} for a model $\mathbb{M}$ is a function from a
			variable to an element in $|\mathbb{M}|$. The syntax $l_{[v \mapsto
			v']}$ defines an environment $l'(x)$ that returns $v'$ when $x=v$
			and returns $l(x)$ otherwise.

		\subsubsection{Satisfiability, Entailment}

			A model $\mathbb{M}$ is said to satisfy a formula $\sigma$ in an environment $l$ when
			\begin{itemize}
			\item $\sigma$ is a relation symbol $R$ and $R[l(a_0) , \ldots , l(a_n)] \in \mathbb{M}$ where $a$ is a set of variables
			\item $\sigma$ is of the form $\neg\alpha$ and $\mathbb{M} \not\models_l \alpha$
			\item $\sigma$ is of the form $\alpha\wedge\beta$ and both $\mathbb{M} \models_l \alpha$ and $\mathbb{M} \models \beta$
			\item $\sigma$ is of the form $\alpha\vee\beta$ and either $\mathbb{M} \models_l \alpha$ or $\mathbb{M} \models \beta$
			\item $\sigma$ is of the form $\alpha\to\beta$ and either $\mathbb{M} \not\models_l \alpha$ or $\mathbb{M} \models \beta$
			\item $\sigma$ is of the form $\forall x : \alpha$  and $\displaystyle\bigwedge_{x' \in |\mathbb{M}|} \mathbb{M} \models_{l[x \mapsto x']} \alpha$
			\item $\sigma$ is of the form $\exists x : \alpha$  and $\displaystyle\bigvee_{x' \in |\mathbb{M}|} \mathbb{M} \models_{l[x \mapsto x']} \alpha$
			\end{itemize}
			This is denoted as $\mathbb{M} \models \sigma$ and read "$\sigma$
			is true in $\mathbb{M}$".  A model $\mathbb{M}$ satisfies a set of
			formulas $\Sigma$ if for every $\sigma$ such that $\sigma \in
			\Sigma$, $\mathbb{M} \models \sigma$.  This is denoted as
			$\mathbb{M} \models \Sigma$ and read "$\mathbb{M}$ is a model of
			$\Sigma$".

			A set of formulas $\Sigma$ is said to \emph{entail} a formula
			$\sigma$ ($\Sigma \models \sigma$) if the set of all models
			satisfied by $\Sigma$ is a subset of the set of all models
			satisfied by $\sigma$.

			The notation used for satisfiability and entailment is very
			similar, in that the operator used ($\models$) is the same, but
			they can be distinguished by the type of left operand.

	\subsection{Homomorphisms}

		A \emph{homomorphism} from $\mathbb{A}$ to $\mathbb{B}$ is a function
		$h: |\mathbb{A}|\to|\mathbb{B}|$ such that, for each relation symbol
		$R$ and tuple $\langle a_0 , \ldots , a_n \rangle$ where $a \subseteq
		|\mathbb{A}|$, $\langle a_0 , \ldots , a_n  \rangle \in R^\mathbb{A}$
		implies $\langle h(a_0) , \ldots , h(a_n) \rangle \in R^\mathbb{B}$.

		A homomorphism $h$ is also a \emph{strong homomorphism} if, for each
		relation symbol $R$ and tuple $\langle a_0 , \ldots , a_n \rangle$
		where $a \subseteq |\mathbb{A}|$, $\langle a_0 , \ldots , a_n  \rangle
		\in R^\mathbb{A}$ if and only if $\langle h(a_0) , \ldots , h(a_n)
		\rangle \in R^\mathbb{B}$.

		The notation $\mathbb{M} \preceq \mathbb{N}$ means that there exists a
		homomorphism $h : \mathbb{M} \to \mathbb{N}$. The identity function is
		a homomorphism from any model $\mathbb{M}$ to itself.  Homomorphisms
		are transitive, so $\mathbb{A} \preceq \mathbb{B} \wedge \mathbb{B}
		\preceq \mathbb{C}$ implies $\mathbb{A} \preceq \mathbb{C}$. However,
		$\mathbb{M} \preceq \mathbb{N} \wedge \mathbb{N} \preceq \mathbb{M}$
		does not imply that $\mathbb{M} = \mathbb{N}$.

		Given models $\mathbb{M}$ and $\mathbb{N}$ where $\mathbb{M} \preceq
		\mathbb{N}$ and a formula in positive-existential form
		\footnote{geometric formulae are implications of positive-existential
		formulae} $\sigma$, if $\mathbb{M} \models \sigma$ then $\mathbb{N}
		\models \sigma$.

		An \emph{isomorphism} is a homomorphism $h : \mathbb{A} \to \mathbb{B}$
		where $h$ is 1:1 and onto and the inverse function $h^{-1} : \mathbb{B}
		\to \mathbb{A}$ is a homomorphism.

		\subsubsection{Significance}

			

		\subsubsection{Minimal Models}

			Minimal models, also called \emph{universal} models, are models for a
			theory with the special property that there exists a homomorphism from
			the minimal model to any other model satisfied by the theory. Minimal
			models have no unnecessary entities or relations and thus display the
			least amount of constraint necessary to satisfy the theory for which
			they are minimal.

			More than one minimal model may exist for a given theory, and not every
			theory must have a minimal model. \textbf{give examples}

		\subsubsection{Relation to the Chase}

			The chase is a function that, when given a gemoetric theory, will
			generate all minimal models for that theory.
