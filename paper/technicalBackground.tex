\section{Technical Background}

	In this paper, possibly ambiguous or uncommon notation will be used and
	thusly must be clearly defined.  Also, topics that are necessary
	prerequisites will be summarized.

	\subsection{Models}

		A \emph{model} $\mathbb{M}$ is a construct that consists of:
		\begin{itemize}
		\item a set, referenced as $|\mathbb{M}|$, called the \emph{universe} or \emph{domain} of $\mathbb{M}$
		\item a set of pairings of a \emph{predicate} and a non-negative integral arity
		\item for each predicate $R$ with arity $k$, a relation $R^\mathbb{M}_k \subseteq |\mathbb{M}|$
		\end{itemize}
		It is important to distinguish the predicate, which is just a symbol,
		from the relation that it refers to when paired with its arity. The
		relation itself is a set of tuples of elements from the universe.

	\subsection{First-order Logic}

		\emph{First-order logic}, also called \emph{predicate logic}, is a
		formal logic system that is an extension of propositional logic. For
		our purposes, this logic system will not contain any constant symbols
		or function symbols, which are commonly included in first-order and
		propositional logic.

		A first-order logic formula is defined inductively by
		\begin{itemize}
		\item if $R$ is a relation symbol of arity $k$ and each of $t_0 , \ldots , t_{k-1}$ is a variable, then $R[t_0,\ldots,t_{k-1}]$ is a formula, specifically an \emph{atomic formula}
		\item $\top$ and $\bot$ are formul{\ae}
		\item if $\alpha$ is a formula then $(\neg\alpha)$ is a formula
		\item if $\alpha$ and $\beta$ are formul{\ae} then $(\alpha\wedge\beta)$ is a formula
		\item if $\alpha$ and $\beta$ are formul{\ae} then $(\alpha\vee\beta)$ is a formula
		\item if $\alpha$ and $\beta$ are formul{\ae} then $(\alpha\to\beta)$ is a formula
		\item if $\alpha$ is a formula and $x$ is a variable then $(\forall x : \alpha)$ is a formula
		\item if $\alpha$ is a formula and $\vec{x}$ is a set of variables of size $k$ then $(\forall \vec{x} : \alpha)$ is $(\forall x_0 \ldots \forall x_{k-1} : \alpha)$
		\item if $\alpha$ is a formula and $x$ is a variable then $(\exists x : \alpha)$ is a formula
		\item if $\alpha$ is a formula and $\vec{x}$ is a set of variables of size $k$ then $(\exists \vec{x} : \alpha)$ is $(\exists x_0 \ldots \exists x_{k-1} : \alpha)$
		\end{itemize}

	\subsection{Positive Existential Form}

		Formulas in \emph{positive existential form} are constrained to using
		only conjunctions ($\wedge$), disjunctions ($\vee$), existential
		quantifications ($\exists$), tautologies ($\top$), contradictions
		($\bot$), and relations to construct logic expressions.

		Though formul{\ae} in positive existential form may at first appear to
		be quite restrictive, there exists some simple logical tricks to allow
		the expression of nearly any first-order logic formula. \textbf{explain these}

	\subsection{Geometric Logic}

		\emph{Geometric logic} formul{\ae} are implicitly universally
		quantified implications of positive existential formul{\ae}. A set of
		geometric logic formul{\ae} is called a \emph{geometric theory}.

		\textbf{explain why GL is useful to us}

	\subsection{Variable Binding}

		The set of free variables in a formula is defined inductively as follows
		\begin{itemize}
		\item any variable occurance in an atomic formula is a free variable
		\item the free variables in $\top$ and $\bot$ are {\o}
		\item the free variables in $\neg\alpha$ are the free variables in $\alpha$
		\item the free variables in $\alpha \wedge \beta$ are the union of the set of free variables in $\alpha$ with the set of free variables in $\beta$
		\item the free variables in $\alpha \vee   \beta$ are the union of the set of free variables in $\alpha$ with the set of free variables in $\beta$
		\item the free variables in $\alpha \to    \beta$ are the union of the set of free variables in $\alpha$ with the set of free variables in $\beta$
		\item the free variables in $\forall x : \alpha$ are the free variables in $\alpha$ that are not $x$
		\item the free variables in $\exists x : \alpha$ are the free variables in $\alpha$ that are not $x$
		\end{itemize}
		A \emph{sentence} is a formula with an empty set of free variables.

	\subsection{Environment}

		An \emph{environment} for a model $\mathbb{M}$ is a function from a
		variable to an element in $|\mathbb{M}|$. The syntax $l_{[v \mapsto
		v']}$ defines an environment $l'(x)$ that returns $v'$ when $x=v$ and
		returns $l(x)$ otherwise.

	\subsection{Satisfiability}

		A model $\mathbb{M}$ is said to satisfy a formula $\sigma$ in an
		environment $l$ when
		\begin{itemize}
		\item $\sigma$ is a relation symbol $R$ and $R[l(a_0) , \ldots , l(a_n)] \in \mathbb{M}$ where $a$ is a set of variables
		\item $\sigma$ is of the form $\neg\alpha$ and $\mathbb{M} \not\models_l \alpha$
		\item $\sigma$ is of the form $\alpha\wedge\beta$ and both $\mathbb{M} \models_l \alpha$ and $\mathbb{M} \models_l \beta$
		\item $\sigma$ is of the form $\alpha\vee\beta$ and either $\mathbb{M} \models_l \alpha$ or $\mathbb{M} \models_l \beta$
		\item $\sigma$ is of the form $\alpha\to\beta$ and either $\mathbb{M} \not\models_l \alpha$ or $\mathbb{M} \models_l \beta$
		\item $\sigma$ is of the form $\forall x : \alpha$  and for every $x' \in |\mathbb{M}|$, $\mathbb{M} \models_{l[x \mapsto x']} \alpha$
		\item $\sigma$ is of the form $\exists x : \alpha$  and for at least one $x' \in |\mathbb{M}|$, $\mathbb{M} \models_{l[x \mapsto x']} \alpha$
		\end{itemize}
		This is denoted as $\mathbb{M} \models_l \sigma$ and read "$\sigma$ is
		true in $\mathbb{M}$". The notation $\mathbb{M} \models \sigma$ (no
		environment specification) means that either, under any environment
		$l$, $\mathbb{M} \models_l \sigma$.

		A model $\mathbb{M}$ satisfies a set of formul{\ae} $\Sigma$ if for every
		$\sigma$ such that $\sigma \in \Sigma$, $\mathbb{M} \models \sigma$.
		This is denoted as $\mathbb{M} \models \Sigma$ and read "$\mathbb{M}$
		is a model of $\Sigma$".

	\subsection{Entailment}

		A set of formulan $\Sigma$ is said to \emph{entail} a formula $\sigma$
		($\Sigma \models \sigma$) if the set of all models satisfied by
		$\Sigma$ is a subset of the set of all models satisfied by $\sigma$.

		The notation used for satisfiability and entailment is very similar, in
		that the operator used ($\models$) is the same, but they can be
		distinguished by the type of left operand.

	\subsection{Homomorphisms}

		A \emph{homomorphism} from $\mathbb{A}$ to $\mathbb{B}$ is a function
		$h: |\mathbb{A}|\to|\mathbb{B}|$ such that, for each relation symbol
		$R$ and tuple $\langle a_0 , \ldots , a_n \rangle$ where $a \subseteq
		|\mathbb{A}|$, $\langle a_0 , \ldots , a_n  \rangle \in R^\mathbb{A}$
		implies $\langle h(a_0) , \ldots , h(a_n) \rangle \in R^\mathbb{B}$.

		A homomorphism $h$ is also a \emph{strong homomorphism} if, for each
		relation symbol $R$ and tuple $\langle a_0 , \ldots , a_n \rangle$
		where $a \subseteq |\mathbb{A}|$, $\langle a_0 , \ldots , a_n  \rangle
		\in R^\mathbb{A}$ if and only if $\langle h(a_0) , \ldots , h(a_n)
		\rangle \in R^\mathbb{B}$.

		The notation $\mathbb{M} \preceq \mathbb{N}$ means that there exists a
		homomorphism $h : \mathbb{M} \to \mathbb{N}$. The identity function is
		a homomorphism from any model $\mathbb{M}$ to itself.  Homomorphisms
		are transitive, so $\mathbb{A} \preceq \mathbb{B} \wedge \mathbb{B}
		\preceq \mathbb{C}$ implies $\mathbb{A} \preceq \mathbb{C}$. However,
		$\mathbb{M} \preceq \mathbb{N} \wedge \mathbb{N} \preceq \mathbb{M}$
		does not imply that $\mathbb{M} = \mathbb{N}$.

		Given models $\mathbb{M}$ and $\mathbb{N}$ where $\mathbb{M} \preceq
		\mathbb{N}$ and a formula in positive-existential form
		\footnote{geometric formul{\ae} are implications of positive-existential
		formul{\ae}} $\sigma$, if $\mathbb{M} \models \sigma$ then $\mathbb{N}
		\models \sigma$.

		A homomorphism $h : \mathbb{A} \to \mathbb{B}$ is also an
		\emph{isomorphism} when $h$ is 1:1 and onto and the inverse function
		$h^{-1} : \mathbb{B} \to \mathbb{A}$ is a homomorphism.

	\subsection{Minimal Models}

		Minimal models, also called \emph{universal} models, are models for a
		theory with the special property that there exists a homomorphism from
		the minimal model to any other model satisfied by the theory. Minimal
		models have no unnecessary entities or relations and thus display the
		least amount of constraint necessary to satisfy the theory for which
		they are minimal.

		More than one minimal model may exist for a given theory, and not every
		theory must have a minimal model. \textbf{give examples}.

		A set of models $M$ is said to be \emph{jointly minimal} for a set of
		formul{\ae} $\Sigma$ when every model $\mu$ such that $\mu \models
		\Sigma$ has a homomorphism from a model $m \in M$ to $\mu$.
