\section{Introduction}

	\subsection{Goals}

		The two main goals of this Major Qualifying Project are:

		\begin{enumerate}
		\item to implement an algorithm known as ``The Chase" accurately
		and with a well-defined, usable interface, and
		\item to demonstrate the chase implementation in a real-world
		application: generating models used in analysis of a specific
		cryptographic protocol
		\end{enumerate}

		Secondary goals include implementing various optimizations and
		integrating the chase implementation into a program that can take
		advantage of the functionality the implementation provides.

	\subsection{The Chase}

		\emph{The chase} is an algorithm used to find jointly universal models
		(Section \ref{sec:technical_background.universal_models}) for a set of
		geometric logic formul{\ae} (Section
		\ref{sec:technical_background.geometric_logic}). Many common real-world
		problems can be expressed as a set of geometric logic formul{\ae}. When
		these problems have an unbounded scope of possible solutions, the chase
		can be used to find the possible solutions that are interesting. This
		allows researchers to go through only models that represent a large set
		of models rather than testing each of the infinite number of models
		separately.


		To generate these jointly universal models, the chase begins with a
		model $\mathbb{M}$ that has an empty domain and no facts. The chase
		goes through each formula $\sigma$ in the geometric theory
		$\mathcal{T}$ such that $\mathbb{M}$ is not a model of $\sigma$. The
		chase is nondeterministic over disjunctions; when a disjunction is
		encountered, only a single disjunct is chosen to be satisfied. The
		chase expands $\mathbb{M}$ until $\mathbb{M}$ is a model of all $\sigma
		\in \mathcal{T}$. Geometric formul{\ae} are implications of
		positive-existential formul{\ae} (Section
		\ref{sec:technical_background.geometric_logic}). Positive-existential
		formul{\ae} have the useful property in that adding elements/facts to a
		model that satisfies one will never cause the model to no longer
		satisfy that formula. Because of this, the chase can keep adding to the
		model until all $\sigma \in \mathcal{T}$ are satisfied. The set of all
		models generated by this process is a jointly universal set for the
		input theory $\mathcal{T}$.

	\subsection{Chase Implementation}

		We will see in Section \ref{sec:implementation} that, in a Haskell
		implementation of the chase, the result of all choices of disjunct in
		any disjunction can be found by forking the chase at each disjunction
		and returning a list containing the concatenation of the lists returned
		by the forks. In this way, we can deterministically calculate a set of
		jointly universal models by finding all of the models returned by
		successful runs of a naturally nondeterministic algorithm.

		The choice of Haskell for an implementation language is very beneficial
		because of Haskell's lazy evaluation. When the chase would run
		infinitely, the program runs infinitely as well, but it will still
		process and return all of the results of the halting runs of the chase.
		Haskell is also commonly used by mathematicians and theoretical
		computer scientists.

	\subsection{Application}

		Cryptographic protocol analysis is the particular application that is
		explored in Section \ref{sec:application.strand_spaces}. In this
		application, protocols are modeled in the strand space formalism. Each
		role of every participant in a legal run of the protocol is modeled as
		a strand. The roles of a special participant that does not obey the
		rules of the protocol are called adversary strands (Section
		\ref{sec:application.the_adversary}). These adversary strands consist
		of a series of nodes that send/receive messages to/from regular strands
		while manipulating those messages.

		Because the positions and actions of adversary strands are variable,
		there exists a large number of possible runs of a single protocol. It
		is prohibitive to test all of these to find if they break assumptions
		made about the properties of the protocol because of the large number
		of possibilities.

		This protocol can be represented as a geometric theory and given to the
		chase to find models. A jointly universal set of models can describe
		nearly all interesting ways that the adversaries can interact with
		regular strands. In the case that there are finitely many models
		returned by the chase, these models can then be analysed in a finite
		amount of time.

	Section \ref{sec:technical_background} can be used as a reference for terms
	discussed in later sections. Appendices \ref{sec:appendix_syntax_table} and
	\ref{sec:appendix_glossary} contain a syntax reference and glossary,
	respectively.
