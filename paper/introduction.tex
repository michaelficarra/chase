\section{Introduction}

	This document details work done for a Major Qualifying Project at Worcester
	Polytechnic Institute by Michael Ficarra in partial fulfillment of the
	requirements for a Bachelor of Science degree.

	\subsection{Goals}

		The two main goals of this Major Qualifying Project are:

		\begin{enumerate}
		\item first, to implement an algorithm known as ``the chase" accurately
		and with a well-defined, usable interface
		\item second, to use the chase implementation for a real-world
		application: generating models used in analysis of a specific protocol
		\end{enumerate}

		Secondary goals include implementing various optimizations and
		integrating the chase implementation into a program that can take
		advantage of the functionality it provides.

	\subsection{The Chase}

		\emph{The chase} is an algorithm used to find jointly minimal models
		for a set of geometric logic formul{\ae}. Many common real-world
		problems can be expressed as a set of geometric logic formul{\ae}.
		When these problems have an unbounded scope of possible solutions, the
		chase can be used to find the possible solutions that are interesting.
		This allows researchers to go through only models that will behave
		uniquely, rather than testing each model separately. Cryptographic
		protocol analysis is a particular application that is explored further.

		To generate these jointly minimal models, the chase begins with a model
		that has an empty domain and no facts. The chase goes through each
		formula in the geometric theory and alters the model so that it
		satisfies that formula.  Geometric formul{\ae} have the useful property
		in that adding elements/facts to a model that satisfies a geometric
		formula will never cause the model to no longer satisfy that formula.
		Because of this, the chase can keep adding to the model until all
		formul{\ae} are satsfied.

		Our specific definition of the chase is nondeterministic over
		disjunctions.  When a disjunction is encountered, a disjunct is chosen
		without prejudice and the chase continues detereministically until it
		encounters another disjunct. We will see that, in a Haskell
		implementation of the chase, both disjuncts can be satisfied by forking
		the chase and returning a list containing the concatentation of the
		lists returned by the forks. In this way, we can deterministically
		calculate the set of jointly minimal models by using a naturally
		nondeterministic algorithm.
