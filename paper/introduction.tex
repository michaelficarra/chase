\section{Introduction}

	\textbf{ remove this? }

	This document details work done for a Major Qualifying Project at Worcester
	Polytechnic Institute by Michael Ficarra in partial fulfillment of the
	requirements for a Bachelor of Science degree in Computer Science.

	\subsection{The Chase}

		\textbf{ this whole section needs some expanding still }

		\emph{The chase} is an algorithm used to find jointly minimal models
		for a set of geometric logic formul{\ae}. Many common real-world
		problems can be expressed as a set of geometric logic formul{\ae}. When
		these problems have an unbounded scope of possible solutions, the chase
		can be used to find the possible solutions that are interesting. This
		allows researchers to go through only models that will behave in a
		unique way, rather than testing each of the infinite number of models
		separately.

		Cryptographic protocol analysis is the particular application that is
		explored further later in this document. In this application, protocols
		are modeled in the strand space formalism. Each participant in a legal
		run of the protocol is called a strand. A special participant that does
		not obey the rules of the protocol is called an adversary strand. These
		adversary strands send/receive messages to/from regular strands, while
		manipulating those messages. Because there is an infinite combination
		of adversary strands, their positions, and the actions they can take,
		it is prohibitive to test if any of these can lead to undesirable
		consequences. This protocol can be represented as a geometric theory
		and passed to the chase to find minimal models. When minimal models are
		found, they can describe all interesting ways that the adversaries can
		interact with regular strands. These models can then be tested in a
		reasonable, finite amount of time.

		To generate these jointly minimal models, the chase begins with a model
		$\mathbb{M}$ that has an empty domain and no facts. The chase goes
		through each formula $\sigma$ in the geometric theory $T$ such that
		$\mathbb{M} \not\models \sigma$ and alters $\mathbb{M}$ so that
		$\mathbb{M} \models \sigma$. Geometric formul{\ae} have the useful
		property in that adding elements/facts to a model that satisfies a
		geometric formula will never cause the model to no longer satisfy that
		formula.  Because of this, the chase can keep adding to the model until
		all formul{\ae} are satisfied. The resulting model is in a set of
		jointly minimal models.

		The chase is nondeterministic over disjunctions. When a disjunction is
		encountered, a disjunct is chosen and the chase continues
		deterministically until it encounters another disjunct.

		We will see that, in a Haskell implementation of the chase, both
		disjuncts can be satisfied by forking the chase and returning a list
		containing the concatenation of the lists returned by the forks. In
		this way, we can deterministically calculate a set of jointly minimal
		models by using a naturally nondeterministic algorithm.

	\subsection{Goals}

		The two main goals of this Major Qualifying Project are:

		\begin{enumerate}
		\item to implement an algorithm known as ``the chase" accurately
		and with a well-defined, usable interface, and
		\item to use the chase implementation for a real-world
		application: generating models used in analysis of a specific protocol
		\end{enumerate}

		Secondary goals include implementing various optimizations and
		integrating the chase implementation into a program that can take
		advantage of the functionality it provides.
